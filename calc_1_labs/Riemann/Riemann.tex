\documentclass{ximera}
\title{Reimann Sums}
\begin{abstract}
\end{abstract}
\begin{document}
\maketitle
\begin{dialogue}
\item[Dylan] Hey Julia, can you help me with this problem?
\item[Julia] Yeah, of course! What do you need?
\item[Dylan] I'm supposed to approximate area under a curve, and I don't really see what to do.
\item[Julia] Actually, that's pretty easy! We'll just use Riemann sums.
\end{dialogue}

\subsubsection{Introduction}
Riemann sums are a method of approximating area under a curve, and they come in three varieties; left, right, and midpoint.

To create Riemann sums, you simply pick a number of desired subintervals, and then evenly divide the interval to produce the desired number. From here, we choose the height of what will be our rectangles differently for each version:
\begin{itemize}
\item{\textbf{Left Riemann Sum:} The height is calculated using the left endpoint of the subinterval.}
\item{\textbf{Right Riemann Sum:} The height is calculated using the right endpoint of the subinterval.}
\item{\textbf{Midpoint Riemann Sum:} The height is calculated using the midpoint of the subinterval.}
\end{itemize}

From here, we simply add the area of each rectangle to produce the area under the curve.

\subsubsection{Increasing, Concave Up}
Consider the function $x^2$ on the interval [1, 6]. Evenly divide the interval into six subintervals.
\begin{question}
Using the given intervals, compute:

The left Riemann sum.

$\answer{}$

The right Riemann sum.

$\answer{}$

The midpoint Riemann sum.

$\answer{}$

Using your CAS, compute the integral numerically. Which of these estimates was most accurate? Show the percent error for each type of Riemann sum.

Most Accurate: $\answer{}$

Left Percent Error: $\answer{}$

Middle Percent Error: $\answer{}$

Right Percent Error: $\answer{}$
\end{question}

\subsubsection{Decreasing, Concave Up}
\begin{question}
Consider the function $x^2$ on the interval [-7, 0]. Evenly divide the interval into seven subintervals.

Using the given intervals, compute:

The left Riemann sum.

$\answer{}$

The right Riemann sum.

$\answer{}$

The midpoint Riemann sum.

$\answer{}$

Using your CAS, compute the integral numerically. Which of these estimates was most accurate? Show the percent error for each type of Riemann sum.

Most Accurate: $\answer{}$

Left Percent Error: $\answer{}$

Middle Percent Error: $\answer{}$

Right Percent Error: $\answer{}$
\end{question}

\subsubsection{Increasing, Concave Down}
Consider the function $\sin(x)$ on the interval [0,$\frac{\pi}{2}$].
\begin{enumerate}
\item{Evenly divide the interval into four subintervals.}
\item{Using your answer to Question 1, compute the left Riemann sum.}
\item{Using your answer to Question 1, compute the right Riemann sum.}
\item{Using your answer to Question 1, compute the midpoint Riemann sum.}
\item{Using your CAS, compute the integral numerically. Which of these estimates was most accurate? Show the percent error for each type of Riemann sum.}
\end{enumerate}

\subsubsection{Decreasing, Concave Down}
Consider the function $\cos(x)$ on the interval [0, $\frac{\pi}{2}$].
\begin{enumerate}
\item{Evenly divide the interval into eight subintervals.}
\item{Using your answer to Question 1, compute the left Riemann sum.}
\item{Using your answer to Question 1, compute the right Riemann sum.}
\item{Using your answer to Question 1, compute the midpoint Riemann sum.}
\item{Using your CAS, compute the integral numerically. Which of these estimates was most accurate? Show the percent error for each type of Riemann sum.}
\end{enumerate}

\begin{dialogue}
\item[Dylan] That's cool! Thanks Julia!
\item[Julia] No problem!
\item[Dylan] I wish I could make the sums more accurate though... some of them are pretty far off.
\item[James] I think if you put your mind to it you could Dylan!
\item[Julia and Dylan] James! You're late to class!
\item[James] Haha no problem, I love helpi... that's not the point! Listen, just use sum notation, and try to make infinitely many subintervals. If you can do that, you'll have an accurate area.
\end{dialogue}
\begin{enumerate}
\item{How would you represent the width of each rectangle when divided into $n$ subintervals? \textit{Hint: Think of the start of the interval as $a$ and the end as $b$.}}
\item{How would you represent the height of each rectangle when divided into $n$ subintervals? \textit{Hint: As $n$ grows larger, will the rectangle be wider than a single point? What does that tell you about the top?}}
\item{Using sigma notation, represent the area under the curve from $i = 1$ to $n$, as $n$ approaches infinity.}
\end{enumerate}

\begin{dialogue}
\item[Julia] Wow, that's just as accurate as asking our computers!
\item[James] That's right Julia! You just found the integral of a function, with just a little guidance!
\item[Julia and Dylan] Wow! Thanks James!
\end{dialogue}

\subsubsection{In Summary}
We've learned a lot about Riemann sums today, and even the formula for a definite integral! So let's recap:
\begin{definition}
A \textbf{Riemann sum} comes in three types, all of which first divide an interval into a number of subintervals:
\begin{enumerate}
\item{\textbf{Left endpoint Riemann sums} use the left endpoint of the subinterval to approximate the area.}
\item{\textbf{Right endpoint Riemann sums} use the right endpoint of the subinterval to approximate the area.}
\item{\textbf{Midpoint Riemann sums} use the midpoint of the subinterval to approximate the area.}
\end{enumerate}
Following this, the area of each rectangle is added to approximate the area under the curve.
\end{definition}

\begin{itemize}
\item{The formula for the definite integral is $$\displaystyle \int_a^b f(x) = \lim_{n\to\infty} \sum_{i=1}^n f(x) \cdot \frac{a-b}{n}$$where $a$ and $b$ are the endpoints of the interval.}
\end{itemize}
\pagebreak
\end{document}
\documentclass{ximera}
\title{Transformations of Functions}
\begin{abstract}
\end{abstract}
\begin{document}
\maketitle
\begin{dialogue}
\item[Julia] Ugh!
\item[Dylan] What's up Julia?
\item[Julia] I have these functions I have to graph, and they're \textit{so} close to functions I know really well, but they're a little bit different and it makes it so I have to calculate a bunch of points before I can confidently graph it!
\item[James] Sounds like you could use some help Julia!
\item[Julia and Dylan] James!
\item[James] There are a ton of ways to transform functions, so let's get going and look at how we can modify our favorite functions!
\end{dialogue}
\section{Introduction}
While you work with many different functions, there are only a few basic types of functions. These include polynomials, rational functions, trigonometric functions, exponential functions, and logarithmic functions. In this lab we will explore different variations on these basic functions called \textbf{transformations}.
\section{Guided Example}
Consider the function $f(x)=x^2$.

\[
\graph{x^2}
\]

\textbf{Answer each question about movement in the form "The graph shifted `direction' X units", where X is the number of units and direction is up, down, left, or right.}

\begin{question}

On the same axis graph $g(x)=x^2+2$, what change happened from $f(x)$ to $g(x)$?

$\answer{The graph shifted up 2 units}$

What can you infer about the function $x^2-2$?

$\answer{The graph shifted down 2 units}$

Graph this function to verify your prediction.

What rule can you write about a general function $f(x)+c$, where $c$ is a constant?
\begin{freeResponse}
\end{freeResponse}
Consider the function $f(x+2)$, or $(x+2)^2$. How do you think this graph will be different from the graph of $f(x)$?
\begin{freeResponse}
\end{freeResponse}
Graph the function $f(x+2)$, was your prediction correct? What can you infer about the function $f(x-2)$? Graph this function to verify your prediction.
\begin{freeResponse}
\end{freeResponse}
What rule can you write about a general function $f(x+c)$ where $c$ is a positive constant(answer in the form "shifts `direction" c units")?
$\answer{shifts left c units}$

Why do you think the graph moves in the direction it does when using the rule you determined in the last question? \textit{Hint: Think about the x-intercept and how it changes when you add or subtract a constant from the x value}
\begin{freeResponse}
\end{freeResponse}
How do you think the graph of $f(x)$ be affected when you multiply the whole function by some constant $c$? Graph the function for the following values of $c=2,\frac{1}{2},-2,\frac{-1}{2}$
\[
\graph{}
\]
\begin{freeResponse}
\end{freeResponse}
Describe what is happening to the function based on the value of $c$, what can you generalize from this? It may be helpful to make a table with the x and y values to understand why this change happens.
\begin{freeResponse}
\end{freeResponse}
\end{question}

\section{On your own}
\begin{question}
Using $g(x) = x^2$ as your base function create a new function that will shift the graph up 4 units, to the right 3 units, reflect it across the x-axis and stretch it vertically by a factor of 2 and graph it below
\[
\graph{}
\]
Graph the function $g(2x)$
\[
\graph{}
\]
What constant does this stretch or compress $x^2$ by?
$\answer{1/c}$
Graph $g(2x+6)$ on the same axis above, what transformation occured? 
\begin{freeResponse}
\end{freeResponse}
Note the following expansion of the general function $f(x)=(ax+b)^2$: $$\displaystyle f(x)=\left(ax+b\right)^2=\left(a\left(x+\frac{b}{a}\right)\right)^2=a^2\left(x+\frac{b}{a}\right)^2$$

From this expansion, how is a function in the form $f(x)=(ax+b)^2$ being shifted and stretched/compressed in terms of $a$ and $b$?
The graph is $\answer{stretched}$ 
\end{question}
\section{In Summary}
Briefly state how the graph of $f(x)=x^n$ changes for each of the following cases.
\begin{question}
$f(x)=cx^n$
\begin{enumerate}
\item When $c>1$ $\answer{}$
\item When $c<1$ $\answer{}$
\item When $0<c<1$ $\answer{}$
\end{enumerate}
$f(x)=(x+c)^n$
\begin{enumerate}
\item When $c>1$ $\answer{}$
\item When $c<1$ $\answer{}$
\end{enumerate}
$f(x)=x^n+c$
\begin{enumerate}
\item When $c>1$ $\answer{}$
\item When $c<1$ $\answer{}$
\end{enumerate}
$f(x)=a(x-b)^n+y$ $\answer{}$
\end{question}


\end{document}
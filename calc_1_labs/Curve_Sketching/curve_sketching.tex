\documentclass{ximera}
\title{Curve Sketching}
\begin{abstract}
\end{abstract}
\begin{document}
\maketitle
\section{Introduction}
\begin{dialogue}
\item[Dylan] Using CAS systems to graph is great and all, but on a test where I don't have a calculator it's so hard to sketch a curve!
\item[James] Well maybe we can use derivatives to figure out properties of the graph so it's easier to sketch!
\item[Dylan] Oh! We'd be able to see where the graph was heading up or down, plus we'd be able to see maxima when the derivative at a point is zero!
\end{dialogue}

\section{Guided Problems}

\begin{question}
Consider the function $f(x)=x^3+12x^2+4$.

Create a number line, and mark the points where the derivative is zero. Between these points, mark the sign of the derivative of any point on that interval (You may do this by hand as it is difficult to format using most CAS systems). An example is shown below.

\begin{image}
\includegraphics{Numberline}
\end{image}

On what interval(s) is the slope positive?

\begin{selectAll}
\choice{$[-8, 0)$}
\choice[correct]{$(-\infty,-8)$}
\choice{$(-\infty, 0]$}
\choice[correct]{$(0,\infty)$}
\end{selectAll}

On what interval(s) is the slope negative?

\begin{selectAll}
\choice{$[-4, \infty)$}
\choice[correct]{$(-8,0)$}
\choice{$(-\infty, 0]$}
\choice{$(0,\infty)$}
\end{selectAll}

Select everything you can predict about the graph of a function at a point where the derivative is 0:

\begin{selectAll}
\choice[correct]{The graph will be flat at that point.}
\choice{The slope will change from positive to negative at the point.}
\choice{The slope will change from negative to positive at the point.}
\choice{The slope will change sign going from one side of the point to the other.}
\end{selectAll}
\begin{feedback}[correct]
If the derivative at a point is zero, we can only tell that the graph flattens out. It is possible that the sign does not change after crossing the point!
\end{feedback}
\end{question}

\begin{dialogue}
\item[Julia] So we have local maxima and minima when the derivative is 0, but what about the graph of $x^3$?
\end{dialogue}

\begin{image}
\includegraphics[scale=.3]{00001}
\end{image}

\begin{dialogue}
\item[Dylan] Hmmm...I guess that means there are three different kinds of critical points! Two when the sign changes and one when it stays the same!
\item[Julia] Wait... what's a critical point?
\item[Dylan] Any point where the derivative is zero or does not exist! Because we know it's important, but we have to check to see what it means with our number line!
\item[James] You guys are still figuring that out? I'm already determining concavity!
\item[Dylan and Julia] Holy cat fur! What's concavity?!
\item[James] A graph is \textit{concave up when its derivative is increasing}, and \textit{concave down when its derivative is decreasing}. The easiest way to tell is to look at the curve and think `Would this hold water?' If it would, it's concave up, and if not, it's concave down!
\item[Dylan and Julia] Wow! Thanks James!
\end{dialogue}
\begin{question}

Now use your CAS to determine the derivative of the function's derivative, $f''(x)$. Points where the second derivative is zero are known as inflection points. What happens at these points?

\begin{multipleChoice}
\choice{They show the function is at extrema as well.}
\choice{They indicate that concavity will remain the same.}
\choice[correct]{They indicate that concavity is changing.}
\choice{They show that the function is at minimum change.}
\end{multipleChoice}

Draw another number line, this time for the second derivative, marking the inflection points. Evaluate $f''(x)$ on a point of each of the intervals created through this marking, and mark the sign. What might this mean in general for changing signs on each side of an inflection point?

\begin{selectAll}
\choice{The graph will be flat at that point.}
\choice{The concavity will change from up to down at the point.}
\choice{The concavity will change from down to up at the point.}
\choice[correct]{The slope will change direction going from one side of the point to the other.}
\end{selectAll}

Where is the graph concave up?
\begin{multipleChoice}
\choice{$[0,1]$}
\choice[correct]{$[-4, \infty)$}
\choice{$(-\infty, 0]$}
\choice{$[0,260]$}
\choice{$[4,260]$}
\end{multipleChoice}

What about concave down?

\begin{multipleChoice}
\choice{$[0,1]$}
\choice{$[-4, \infty)$}
\choice[correct]{$(-\infty, -4]$}
\choice{$[0,260]$}
\choice{$[4,260]$}
\end{multipleChoice}

Inflection points are given as ordered pairs. Evaluate each inflection point you found using $f(x)$ to determine the ordered pairs, then select them below.
\begin{selectAll}
\choice[correct]{$(-4,132)$}
\choice{$(0, 4)$}
\choice{$(-12.028, 0)$}
\choice{$(-8,260)$}
\end{selectAll}

Based on what you've done until now, sketch the graph yourself. When you're done, click below to see the graph.

\begin{multipleChoice}
\choice[correct]{I'm Done! Show me the graph!}
\end{multipleChoice}
\begin{feedback}
\[
\graph{x^3+12x^2+4}
\]
\end{feedback}
\end{question}

\section{Matching Graphs}
\begin{dialogue}
\item[Dylan]So if I wanted to match a graph with the graphs of its first and second derivative I can do that now!
\item[Julia] Wait, really?? How?
\item[James] Well the $y$ value of $f'(x)$ corresponds to the slope of $f(x)$, and the $y$ value of $f''(x)$ corresponds to the slope of $f'(x)$ and is related to the concavity of $f(x)$.
\item[Julia] So we can match the graphs based on how all that information relates!
\item[Dylan] Exactly, let's try it!
\end{dialogue}
\begin{question}
\choice{\begin{image}
\includegraphics{matching}
\end{image}}
\choice{something else}
\end{question}
Match each graph to its first and second derivative, create a table to organize by $f(x)$,$f'(x)$, and $f''(x)$.

\begin{image}
\centering
\includegraphics{matching}
\end{image}

\section{On Your Own - Need Answers}
Now, for the function $$x^5-5x^4+5x^3+5x^2-6x-1$$. Find:
\begin{itemize}
\item{Extrema}
\item{Critical Points}
\item{Inflection Points}
\item{Concavity}
\end{itemize}


\section{In Summary}
In this lab, you've covered quite a bit. To help organize everything, we've made the following table for you.


\begin{tabular}{| l | p{7.5cm} |}
\hline
First Derivative Test & When $f'(x) = 0$: \begin{enumerate}
\item{A maximum occurs if at this point, the sign of the derivative changes from positive to negative.}
\item{A minimum occurs at this point if the sign of the derivative changes from negative to positive.}
\item{An inflection point occurs at this point if the sign of the derivative stays the same.}
\end{enumerate}
When $f'(x)$ does not exist:
\begin{enumerate}
\item{If $x$ is in the domain of $f(x)$, follow the same steps as when $f'(x)=0\text{.}$}
\item{If $x$ is not in the domain, then $x$ is not a critical point.}
\end{enumerate}\\
\hline
Second Derivative Test &  \hspace{5mm}If $f'(x)=0\text{,}$ \begin{enumerate}
\item{A local maximum occurs if $f''(x)>0$.}
\item{A local minimum occurs if $f''(x)<0$.}
\item{The test fails if $f''(x)=0$.}
\end{enumerate}\\ \hline
\end{tabular}
\end{document}
%\pagebreak
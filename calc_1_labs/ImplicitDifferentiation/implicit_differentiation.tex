\documentclass{ximera}
\title{Implicit Differentiation - Finish solutions}
\begin{abstract}
\end{abstract}
\begin{document}
\maketitle
\begin{dialogue}
\item[Dylan] Woah! What's up with this?
\item[Julia] I didn't know functions were explicit!
\item[Dylan] The $x$ and $y$ are on the same side of the equation! I can't deal with this.
\item[James] Functions can be explicit or implicit! And it not the way you're thinking Julia...
\end{dialogue}
\section{Introduction}
So far we have dealt only with explicitly defined functions, where $y=f(x)$.  Functions where there are both $x$ and $y$ on one side or both sides of the equation are called \textbf{implicit functions}.
\section{Guided Example}
\begin{question}
Which of the following equations are defined implicitly?
\begin{selectAll}
\choice{$$y=x^2+5x-7$$}
\choice{$$y=\sin(x)$$}
\choice[correct]{$$x^2+y^2=1$$}
\choice{$$y=\sqrt(x-3)$$}
\choice[correct]{$x^2y^3+y = 5x+8y$}
\end{selectAll}
\end{question}
\begin{question}
Graph the following implicitly defined function below, $$x^2+y^2=1$$
\[
\graph{}
\]
Now, in the following sage cell, solve the function for y. For help using the solve command refer to the \href{http://doc.sagemath.org/html/en/tutorial/tour_algebra.html#solving-equations}{documentation} here.
\begin{onlineOnly}
\begin{sageCell}
x,y = var("x, y")
#eqn = x**2+y**2==1, this sets eqn to the unit circle
#use the solve command to solve eqn for y
\end{sageCell}
\end{onlineOnly}
Graph the two explicit equations on the same axis below.
\[
\graph{}
\]
Which of the following are true?
\begin{selectAll}
\choice{$$x^2+y^2=1$$ is a function}
\choice[correct]{$$-\sqrt{1-x^2}$$ is a function}
\choice[correct]{$$\sqrt{1-x^2}$$ is a function}
\end{selectAll}

\end{question}

\begin{question}
Using the function(s) you found, find the slope of the tangent lines at a point with two values on your graph.
\begin{freeResponse}
\end{freeResponse}
\end{question}
\setcounter{problem}{0}
\section{Implicit Differentiation Using Substitution}
Consider the equation $-x^2 \cdot y^3+y^5-32 = 0$.
\begin{question}
Using the method shown in the previous section, evaluate the function for $y$.
\begin{onlineOnly}
\begin{sageCell}
x,y = var("x, y")
\end{sageCell}
\end{onlineOnly}
Does this equation look easy to differentiate?

$\answer{No}$

Instead, let's treat our equation as an expression - because it equals zero, we don't have to worry about moving anything over. Now consider $y$ as $y(x)$, a function of $x$, and differentiate with respect to $x$. Each $y$ term will gain $\frac{dy}{dx}$. Then, set the expression equal to zero, and solve for $\frac{dy}{dx}$. What does this represent?
\begin{freeResponse}
\end{freeResponse}
\end{question}

\begin{question}
Using your result in the previous section, evaluate $\frac{dy}{dx}$ at $x = 3$ and $x = 7$.

$x = 3: \answer{}$

$x = 7: \answer{}$
\end{question}
\begin{question}
\begin{onlineOnly}
\begin{sageCell}
var("x, y")

\end{sageCell}
\end{onlineOnly}

Now use Sage Math to find the slope of  $\sin(x^2)=\cos(xy^2)$ at any point. \href{http://doc.sagemath.org/html/en/tutorial/tour_algebra.html#differentiation-integration-etc}{Look here for information on implicit differentiation in Sage}

$\answer{}$
\end{question}
\section{Perpendicular at a Point}
\begin{dialogue}
\item[Julia] Wow, implicit differentiation is rough.
\item[Dylan] You're telling me... I've been doing this for hours! I wish we could at least do a little more with it if I have to learn it.
\item[James] Did I hear that you guys want to know more about using implicit differentiation?
\item[Julia and Dylan] James! Tell us more!
\item[James] Alright guys, you can use implicit differentiation with implicit functions to tell if two functions are perpendicular at a point!
\item[Julia] But how?
\item[Dylan] Yeah, I don't see how that helps.
\item[James] It's easy - all we have to do is see if the tangent lines are perpendicular at that point, and if they are, then so are the curves!
\end{dialogue}
\begin{question}
Graph $3x - 2y + x^3-x^2y = 0$ and $x^2 - 2x + y^2 - 3y = 0$ on the same set of axes.
\[
\graph{}
\]

Do they look perpendicular anywhere?
$\answer{Yes}$
\end{question}
\begin{question}
\begin{hint}
Remember, two perpendicular lines will have their slopes be the negative inverse of one another.
\end{hint}
Prove the two curves are (or are not) perpendicular at the origin.
\begin{freeResponse}
\end{freeResponse}
\end{question}


\section{In Summary}
There are two main methods to solve implicit equations 
\begin{enumerate}
\item{Solve for $y$ and then differentiate.}
\item{Treat $y$ as $y(x)$ and differentiate for $x$, eventually solving for $\frac{dy}{dx}$ to give the value of the derivative at any point.}
\end{enumerate}
\pagebreak
\end{document}

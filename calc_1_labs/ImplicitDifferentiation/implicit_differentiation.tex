\documentclass{ximera}
\title{Implicit Differentiation}
\begin{abstract}
\end{abstract}
\begin{document}
\maketitle
\begin{javascript}
 caseInsensitive = function(a,b) {
    return a.toLowerCase() == b.toLowerCase();
  };
\end{javascript}
\begin{dialogue}
\item[Dylan] Woah! What's up with this?
\item[Julia] I didn't know functions were explicit!
\item[Dylan] The $x$ and $y$ are on the same side of the equation! I can't deal with this.
\item[James] Functions can be explicit or implicit! And it not the way you're thinking Julia...
\end{dialogue}
\section{Introduction}
So far we have dealt only with explicitly defined functions, where $y=f(x)$. Here $y$ is dependent variable and it is given in terms of the independent variable $x$. Functions given in terms of both independent and dependent variables are called \textit{implicit} functions.
\section{Guided Example}
\begin{question}
Which of the following equations defined $y$ as a function of $x$ implicitly?
\begin{selectAll}
\choice{$$y=x^2+5x-7$$}
\choice{$$y=\sin(x)$$}
\choice[correct]{$$x^2+y^2=1$$}
\choice{$$y=\sqrt{(x-3)}$$}
\choice[correct]{$x^2y^3+y = 5x+8y$}
\end{selectAll}
\end{question}
Graph the curve defined by this equation: $$x^2+y^2=1$$
\[
\graph{}
\]
Now, in the following Sage cell, solve for $y$. For help using the solve command refer to the \href{http://doc.sagemath.org/html/en/tutorial/tour_algebra.html#solving-equations}{documentation} here.
\begin{onlineOnly}
\begin{sageCell}
x,y = var("x, y")
#eqn = x^2+y^2==1, this sets eqn to the unit circle
#use the solve command to solve eqn for y
\end{sageCell}
\end{onlineOnly}
Graph the two explicit equations on the same axis below.
\[
\graph{}
\]
\begin{question}
Which of the following are true?
\begin{selectAll}
\choice{$x^2+y^2=1$ is a function of $x$.}
\choice[correct]{$y=-\sqrt{1-x^2}$ is a function of $x$.}
\choice[correct]{$y=\sqrt{1-x^2}$ is a function of $x$.}
\end{selectAll}
\end{question}

\begin{question}
\begin{onlineOnly}
\begin{sageCell}

\end{sageCell}
\end{onlineOnly}
Using the functions you found, differentiate to find the slope of the tangent lines at the point $\big( \left(\frac{\sqrt{2}}{2}\right),\left(\frac{\sqrt{2}}{2}\right) \big)$. You may do this in the above Sage cell or by hand.

$\answer{-1}$

\end{question}
Unfortunately not all implicit equations can be easily solved for $y$, which is why we use implicit differentiation!
\begin{explanation}
Starting with
$$x^2 + y^2 = 1$$
we first differentiate each term using $\frac{d}{dx}$
$$\frac{d}{dx}x^2+\frac{d}{dx}y^2 = \frac{d}{dx} 1$$
You can already fill in 2 of the terms
$$ \answer{2x}+ \frac{d}{dx}y^2 = \answer{0}$$
For the term $\frac{d}{dx} y^2$ you can imagine $y = f(x)$, and hence by the chain rule
\begin{align*}
 \frac{d}{dx} y^2 &= \frac{d}{dx}(f(x))^2 \\
 &= 2\cdot f(x) \cdot f'(x)  \\
 &= 2y\frac{dy}{dx} \\
\end{align*}
Thus we have the following:
 $$2x + 2y\frac{dy}{dx} =0$$
Solving for $\frac{dy}{dx}$ we get 
$$\answer{\frac{-x}{y}}$$
\end{explanation}
\begin{question}
Use the equation obtained from the above explanation to find $\frac{dy}{dx}$ at $\left(\frac{\sqrt{2}}{2}\right),\left(\frac{\sqrt{2}}{2}\right)$ 
$\answer{-1}$
\begin{feedback}
Using both methods you can obtain the same answer, but for many equations the first method is much more work!
\end{feedback}
\end{question}
\begin{question}
We can fairly easily use Sage to do this process for us, to illustrate the process evaluate the following Sage cell.
\begin{onlineOnly}
\begin{sageCell}
var('x,y')
y(x)=function('y')(x)
eq=x^2+y^2==1
eq.substitute(y =y(x))
diff(eq,x)
solve(diff(eq,x),diff(y(x)))
\end{sageCell}
\end{onlineOnly}
What did you get as output from your Sage cell? (copy just the answer portion after the ``==" and before the ``]")
$\answer{-x/y(x)}$
\begin{feedback}
Notice that Sage uses y(x) for y in the output.
\end{feedback}
\end{question}
\section{On Your Own}
Consider the equation $y^4+xy=x^3-x+2$. Using the following Sage cell implicitly differentiate to find $\frac{dy}{dx}$ using the same commands as shown in the previous question.
\begin{onlineOnly}
\begin{sageCell}
var('x,y')
\end{sageCell}
\end{onlineOnly}
\begin{question}
What did you get as output from your Sage cell? (copy just the answer portion after the ``==" and before the ``]")

$\answer[format=string]{(3*x^2 - y(x) - 1)/(4*y(x)^3 + x)}$
\end{question}
\begin{question}
Using your result in the previous section, evaluate $\frac{dy}{dx}$ at the point $(1,1)$.
$\answer{1/5}$
\end{question}
\begin{question}
Now use Sage Math again to find $\frac{dy}{dx}$ for  $\sin(x^2)=\cos(xy^2)$, copy your answer in the same way as indicated in the previous section.
\begin{onlineOnly}
\begin{sageCell}
var('x,y')
\end{sageCell}
\end{onlineOnly}
$\answer[format=string]{-1/2*(sin(x*y(x)^2)*y(x)^2 + 2*x*cos(x^2))/(x*sin(x*y(x)^2)*y(x))}$
\end{question}
\section{Perpendicular at a Point}
\begin{dialogue}
\item[Julia] Wow, implicit differentiation is rough.
\item[Dylan] You're telling me... I've been doing this for hours! I wish we could at least do a little more with it if I have to learn it.
\item[James] Did I hear that you guys want to know more about using implicit differentiation?
\item[Julia and Dylan] James! Tell us more!
\item[James] Alright guys, you can use implicit differentiation with implicit functions to tell if two functions are perpendicular at a point!
\item[Julia] But how?
\item[Dylan] Yeah, I don't see how that helps.
\item[James] It's easy - all we have to do is see if the tangent lines are perpendicular at that point, and if they are, then so are the curves!
\end{dialogue}
\begin{question}
Graph $3x - 2y + x^3-x^2y = 0$ and $x^2 - 2x + y^2 - 3y = 0$ on the same set of axes.
\[
\graph{}
\]

Do they look perpendicular anywhere?
$\answer[format=string,validator=caseInsensitive]{Yes}$
\end{question}
\begin{question}
\begin{hint}
To show two lines are perpendicular you must show that the slope of one is the opposite inverse of the other
\end{hint}
Show the two curves are (or are not) perpendicular at the origin. You can do this in the Sage cell provided or by hand.
\begin{onlineOnly}
\begin{sageCell}

\end{sageCell}
\end{onlineOnly}
Slope of $3x - 2y + x^3-x^2y = 0$ at the origin 

$\answer{3/2}$

Slope of $x^2 - 2x + y^2 - 3y = 0$ at the origin 

$\answer{-2/3}$

Are the lines perpendicular at the origin?



\end{question}
\section{In Summary}
There are two main methods to differentiate implicit equations
\begin{enumerate}
\item{Solve for $y$ and then differentiate.}
\item{Treat $y$ as $y(x)$ and differentiate with respect to the variable $x$, eventually solving for $\frac{dy}{dx}$ to give the value of the derivative at any point $(x,y)$.}
\end{enumerate}
\pagebreak
\end{document}

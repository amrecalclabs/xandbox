\documentclass{ximera}
%\usepackage{todonotes}

\newcommand{\todo}{}

\usepackage{esint} % for \oiint
\graphicspath{
{./}
{functionsOfSeveralVariables/}
{normalVectors/}
{lagrangeMultipliers/}
{vectorFields/}
{greensTheorem/}
{shapeOfThingsToCome/}
}


\usepackage{tkz-euclide}
\tikzset{>=stealth} %% cool arrow head
\tikzset{shorten <>/.style={ shorten >=#1, shorten <=#1 } } %% allows shorter vectors

\usetikzlibrary{backgrounds} %% for boxes around graphs
\usetikzlibrary{shapes,positioning}  %% Clouds and stars
\usetikzlibrary{matrix} %% for matrix
\usepgfplotslibrary{polar} %% for polar plots
\usetkzobj{all}
\usepackage[makeroom]{cancel} %% for strike outs
%\usepackage{mathtools} %% for pretty underbrace % Breaks Ximera
\usepackage{multicol}
\usepackage{pgffor} %% required for integral for loops


%% http://tex.stackexchange.com/questions/66490/drawing-a-tikz-arc-specifying-the-center
%% Draws beach ball
\tikzset{pics/carc/.style args={#1:#2:#3}{code={\draw[pic actions] (#1:#3) arc(#1:#2:#3);}}}



\usepackage{array}
\setlength{\extrarowheight}{+.1cm}   
\newdimen\digitwidth
\settowidth\digitwidth{9}
\def\divrule#1#2{
\noalign{\moveright#1\digitwidth
\vbox{\hrule width#2\digitwidth}}}





\newcommand{\RR}{\mathbb R}
\newcommand{\R}{\mathbb R}
\newcommand{\N}{\mathbb N}
\newcommand{\Z}{\mathbb Z}


%\renewcommand{\d}{\,d\!}
\renewcommand{\d}{\mathop{}\!d}
\newcommand{\dd}[2][]{\frac{\d #1}{\d #2}}
\newcommand{\pp}[2][]{\frac{\partial #1}{\partial #2}}
\renewcommand{\l}{\ell}
\newcommand{\ddx}{\frac{d}{\d x}}

\newcommand{\zeroOverZero}{\ensuremath{\boldsymbol{\tfrac{0}{0}}}}
\newcommand{\inftyOverInfty}{\ensuremath{\boldsymbol{\tfrac{\infty}{\infty}}}}
\newcommand{\zeroOverInfty}{\ensuremath{\boldsymbol{\tfrac{0}{\infty}}}}
\newcommand{\zeroTimesInfty}
{\ensuremath{\small\boldsymbol{0\cdot \infty}}}
\newcommand{\inftyMinusInfty}{\ensuremath{\small\boldsymbol{\infty - \infty}}}
\newcommand{\oneToInfty}{\ensuremath{\boldsymbol{1^\infty}}}
\newcommand{\zeroToZero}{\ensuremath{\boldsymbol{0^0}}}
\newcommand{\inftyToZero}{\ensuremath{\boldsymbol{\infty^0}}}



\newcommand{\numOverZero}{\ensuremath{\boldsymbol{\tfrac{\#}{0}}}}
\newcommand{\dfn}{\textbf}
%\newcommand{\unit}{\,\mathrm}
\newcommand{\unit}{\mathop{}\!\mathrm}
\newcommand{\eval}[1]{\bigg[ #1 \bigg]}
\newcommand{\seq}[1]{\left( #1 \right)}
\renewcommand{\epsilon}{\varepsilon}
\renewcommand{\phi}{\varphi}


\renewcommand{\iff}{\Leftrightarrow}

\DeclareMathOperator{\arccot}{arccot}
\DeclareMathOperator{\arcsec}{arcsec}
\DeclareMathOperator{\arccsc}{arccsc}
\DeclareMathOperator{\si}{Si}
\DeclareMathOperator{\proj}{\vec{proj}}
\DeclareMathOperator{\scal}{scal}
\DeclareMathOperator{\sign}{sign}


%% \newcommand{\tightoverset}[2]{% for arrow vec
%%   \mathop{#2}\limits^{\vbox to -.5ex{\kern-0.75ex\hbox{$#1$}\vss}}}
\newcommand{\arrowvec}{\overrightarrow}
%\renewcommand{\vec}[1]{\arrowvec{\mathbf{#1}}}
\renewcommand{\vec}{\mathbf}
\newcommand{\veci}{{\boldsymbol{\hat{\imath}}}}
\newcommand{\vecj}{{\boldsymbol{\hat{\jmath}}}}
\newcommand{\veck}{{\boldsymbol{\hat{k}}}}
\newcommand{\vecl}{\boldsymbol{\l}}
\newcommand{\uvec}[1]{\mathbf{\hat{#1}}}
\newcommand{\utan}{\mathbf{\hat{t}}}
\newcommand{\unormal}{\mathbf{\hat{n}}}
\newcommand{\ubinormal}{\mathbf{\hat{b}}}

\newcommand{\dotp}{\bullet}
\newcommand{\cross}{\boldsymbol\times}
\newcommand{\grad}{\boldsymbol\nabla}
\newcommand{\divergence}{\grad\dotp}
\newcommand{\curl}{\grad\cross}
%\DeclareMathOperator{\divergence}{divergence}
%\DeclareMathOperator{\curl}[1]{\grad\cross #1}
\newcommand{\lto}{\mathop{\longrightarrow\,}\limits}

\renewcommand{\bar}{\overline}

\colorlet{textColor}{black} 
\colorlet{background}{white}
\colorlet{penColor}{blue!50!black} % Color of a curve in a plot
\colorlet{penColor2}{red!50!black}% Color of a curve in a plot
\colorlet{penColor3}{red!50!blue} % Color of a curve in a plot
\colorlet{penColor4}{green!50!black} % Color of a curve in a plot
\colorlet{penColor5}{orange!80!black} % Color of a curve in a plot
\colorlet{penColor6}{yellow!70!black} % Color of a curve in a plot
\colorlet{fill1}{penColor!20} % Color of fill in a plot
\colorlet{fill2}{penColor2!20} % Color of fill in a plot
\colorlet{fillp}{fill1} % Color of positive area
\colorlet{filln}{penColor2!20} % Color of negative area
\colorlet{fill3}{penColor3!20} % Fill
\colorlet{fill4}{penColor4!20} % Fill
\colorlet{fill5}{penColor5!20} % Fill
\colorlet{gridColor}{gray!50} % Color of grid in a plot

\newcommand{\surfaceColor}{violet}
\newcommand{\surfaceColorTwo}{redyellow}
\newcommand{\sliceColor}{greenyellow}




\pgfmathdeclarefunction{gauss}{2}{% gives gaussian
  \pgfmathparse{1/(#2*sqrt(2*pi))*exp(-((x-#1)^2)/(2*#2^2))}%
}


%%%%%%%%%%%%%
%% Vectors
%%%%%%%%%%%%%

%% Simple horiz vectors
\renewcommand{\vector}[1]{\left\langle #1\right\rangle}


%% %% Complex Horiz Vectors with angle brackets
%% \makeatletter
%% \renewcommand{\vector}[2][ , ]{\left\langle%
%%   \def\nextitem{\def\nextitem{#1}}%
%%   \@for \el:=#2\do{\nextitem\el}\right\rangle%
%% }
%% \makeatother

%% %% Vertical Vectors
%% \def\vector#1{\begin{bmatrix}\vecListA#1,,\end{bmatrix}}
%% \def\vecListA#1,{\if,#1,\else #1\cr \expandafter \vecListA \fi}

%%%%%%%%%%%%%
%% End of vectors
%%%%%%%%%%%%%

%\newcommand{\fullwidth}{}
%\newcommand{\normalwidth}{}



%% makes a snazzy t-chart for evaluating functions
%\newenvironment{tchart}{\rowcolors{2}{}{background!90!textColor}\array}{\endarray}

%%This is to help with formatting on future title pages.
\newenvironment{sectionOutcomes}{}{} 



%% Flowchart stuff
%\tikzstyle{startstop} = [rectangle, rounded corners, minimum width=3cm, minimum height=1cm,text centered, draw=black]
%\tikzstyle{question} = [rectangle, minimum width=3cm, minimum height=1cm, text centered, draw=black]
%\tikzstyle{decision} = [trapezium, trapezium left angle=70, trapezium right angle=110, minimum width=3cm, minimum height=1cm, text centered, draw=black]
%\tikzstyle{question} = [rectangle, rounded corners, minimum width=3cm, minimum height=1cm,text centered, draw=black]
%\tikzstyle{process} = [rectangle, minimum width=3cm, minimum height=1cm, text centered, draw=black]
%\tikzstyle{decision} = [trapezium, trapezium left angle=70, trapezium right angle=110, minimum width=3cm, minimum height=1cm, text centered, draw=black]

\title{Differentiation Rules! Again!}
\begin{abstract}
\end{abstract}
\begin{document}
\maketitle
\begin{javascript}
 caseInsensitive = function(a,b) {
    return a.toLowerCase() == b.toLowerCase();
  };
\end{javascript}
\begin{dialogue}
\item[Julia] You know, some of those rules we learned were pretty useful, but some of these derivatives still suck! There \textbf{HAS} to be a better way!
\item[Dylan] I'm sure there is, and I'm sure I know who could help us!
\item[James] Did I hear my name?
\item[Dylan] Not yet!
\item[Julia] James!
\item[James] There are more rules for differentiation that can make your life just a little bit easier!
\end{dialogue}
\section{The Product Rule}
\begin{dialogue}
\item[James] From the last time we did this, what rule do you think would exist for the product of two functions?
\item[Julia] Well, last time we added or subtracted the derivative of both functions, so I bet we multiply the derivative of both!
\item[Dylan] Let's check!
\end{dialogue}
Consider the functions $f(x) = 2x$ and $g(x) = 3x^3 + x^2$.
\[
\graph{2x, 3x^3+x^2}
\]
\begin{question}
Use Julia's guess to find the derivative of $f(x) \cdot g(x)$.

$\answer{18x^2+4x}$

\begin{definition}
  The \textbf{derivative} of $f(x)$ at $a$ is defined by the following limit:
  \[
  \eval{\frac{d}{dx} f(x)}_{x=a} = \lim_{h\to 0} \frac{f(a+h) - f(a)}{h}.
  \]
\end{definition}

Use the limit definition of the derivative to find the derivative of $f(x) \cdot g(x)$.

$\answer{24x^3+6x^2}$

Was Julia right?

$\answer[format=string,validator=caseInsensitive]{No}$
\end{question}
\begin{dialogue}
\item[Julia] Darn! It didn't work!
\item[Dylan] It must be a little harder than that...
\item[James] That's right Dylan, but it is easier than the limit definition! All we have to do is use $$\frac{d}{dx}\left(f(x)\cdot g(x)\right)= f(x)\cdot g'(x) + f'(x)\cdot g(x)\text{.}$$ This is called the \textbf{Product Rule}.
\end{dialogue}
\begin{question}
Using the Product Rule, differentiate the products of the following functions:

$f(x) = 6x^3$, $g(x) = 7x^4$

$\answer{294x^6}$

$f(x) = \cos(x)+4x$, $g(x) = 3x^2+x$

$\answer{3x^2\sin(x)+36x^2+x\sin(x)+6x\cos(x)+8x+\cos(x)}$

$f(x) = x^2$, $g(x) = 3x^3-3x$

$\answer{15x^4-9x^2}$

$f(x) = x^7$, $g(x) = 2x^{32}$

$\answer{78x^{38}}$

\end{question}

\section{The Quotient Rule}
\begin{dialogue}
\item[Dylan] Wow! That's gonna save a ton of time with products! Is there anything like it we can do with quotients?
\item[James] There is! It's even called \textbf{the Quotient Rule}!
\item[Julia] I bet it's a pain too though, just like the product rule.
\item[James] Well, why don't you try using your intuition first rather than guessing?
\item[Dylan] Alright, well, I guess I would divide the derivative of the numerator by the derivative of the denominator.
\end{dialogue}
\begin{question}
Consider the functions $f(x) = x^3$ and $g(x) = \cos(x)$.
\[
\graph{x^3, cos(x)}
\]
Use Dylan's guess to find the derivative of $\frac{f(x)}{g(x)}$.

$\answer{3x^2/\sin(x)}$

Use the limit definition of the derivative to find the derivative of $\frac{f(x)}{g(x)}$.

$\answer{(\cos(x)3x^2-x^3\sin(x))/\cos(x)^2}$

Was Dylan right?

$\answer[format=string,validator=caseInsensitive]{No}$
\end{question}
\begin{dialogue}
\item[Julia] I knew it! It's never that easy!
\item[James] Now calm down Julia, this rule is worse than the last one, but it's much better than going through by the limit definition:  $$\frac{d}{dx}\left( \frac{f(x)}{g(x)}\right) = \frac{f'(x)g(x)-f(x)g'(x)}{g(x)^2}\text{.}$$
\end{dialogue}
\begin{question}
Using the Quotient Rule, differentiate the products of the following functions:

$f(x) = \sin(x)+x^2$, $g(x) = 3x^3+x$

$\answer{(-3x^4+3x^3\cos(x)+x^2-9x^2\sin(x)-x\cos(x)-\sin(x))/(x^2(3x^2+1)^2)}$

$f(x) = \cos(x)+4x$, $g(x) = 3x^2+x$

$\answer{(-x(12x+(3x+1)\sin(x))+(6x+1)\cos(x))/(x^2(3x+1)^2)}$

$f(x) = x^2$, $g(x) = 3x^3-3x$

$\answer{(-x^2-1)/(3(x^2-1)^2)}$

$f(x) = x^7$, $g(x) = 2x^{32}$

$\answer{-25/(2x^{26})}$
\end{question}
\section{The Chain Rule}
\begin{dialogue}
\item[James] There's one last rule to learn today; the \textbf{Chain Rule}.
\item[Dylan] That rule sounds pretty cool! When do we use it though? I thought we already covered the functions we need to know...
\item[Julia] Yeah, what else is there?
\item[James] We use the chain rule in composition of functions, like when we have $\sin(2x)$ - $2x$ is a function, and so is $\sin(x)$
\item[Julia] And how bad is the rule?
\item[James] This one is a little more tricky - $$\frac{d}{dx}f(g(x)) = f'(g(x))*g'(x)\text{.}$$
\item[Dylan and Julia] That's so gross.
\item[James] Well, let's give it a try and see if you like it more than the limit definition!
\end{dialogue}
\begin{question}
Consider $f(x) = \cos(x)$ and $g(x) = 2x$
\[
\graph{cos(x), 2x}
\]

Using the limit definition of derivative, evaluate the derivative of $f(g(x))$.

$\answer{-2\sin(2x)}$

Now, evaluate the same limit using the chain rule. Was it any better?

$\answer[format=string,validator=caseInsensitive]{Yes}$
\end{question}
\begin{question}

Using the Chain Rule, differentiate the compositions $f(g(x))$ for the following functions:

$f(x) = 3x+x^2$, $g(x) = x^4+7x$

$\answer{8x^7+70x^4+12x^3+98x+21}$

$f(x) = \cos(x)$, $g(x) = \sin(x)$

$\answer{-\cos(x)\sin(\sin(x))}$

$f(x) = x^2-5x$, $g(x) = \sqrt{x+3}$

$\answer{1-5/(2\sqrt(x+3))}$

$f(x) = x^7$, $g(x) = \sin(x)-x^3+3$

$\answer{}$
\end{question}

\begin{question}

Using the Chain Rule, differentiate the compositions $g(f(x))$ for the following functions:

$f(x) = 3x+x^2$, $g(x) = x^4+7x$

$\answer{}$

$f(x) = \cos(x)$, $g(x) = \sin(x)$

$\answer{}$

$f(x) = x^2-5x$, $g(x) = \sqrt{x+3}$

$\answer{}$

$f(x) = x^7$, $g(x) = \sin(x)-x^3+3$

$\answer{}$
\end{question}
\end{document}

\documentclass{ximera}
\title{Differentiation Rules! Again!}
\begin{abstract}
\end{abstract}
\begin{document}
\maketitle
\begin{dialogue}
\item[Julia] You know, some of those rules we learned were pretty useful, but some of these derivatives still suck! There \textbf{HAS} to be a better way!
\item[Dylan] I'm sure there is, and I'm sure I know who could help us!
\item[James] Did I hear my name?
\item[Dylan] Not yet!
\item[Julia] James!
\item[James] There are more rules for differentiation that can make your life just a little bit easier!
\end{dialogue}
\subsubsection{The Product Rule}
\begin{dialogue}
\item[James] From the last time we did this, what rule do you think would exist for the product of two functions?
\item[Julia] Well, last time we added or subtracted the derivative of both functions, so I bet we multiply the derivative of both!
\item[Dylan] Let's check!
\end{dialogue}
Consider the functions $f(x) = 2x$ and $g(x) = 3x^3 + x^2$.
\[
\graph{2x, 3x^3+x^2}
\]
\begin{question}
Use Julia's guess to find the derivative of $f(x) * g(x)$.
$\answer{18x^2+4x}$
Use the limit definition of the derivative to find the derivative of $f(x) * g(x)$.

$\answer{}$

Was Julia right?

$\answer{No}$
\end{question}
\begin{dialogue}
\item[Julia] Darn! It didn't work!
\item[Dylan] It must be a little harder than that...
\item[James] That's right Dylan, but it is easier than the limit definition! All we have to do is use $$\frac{d}{dx} f(x)*g(x) = f(x)*g'(x) + f'(x)*g(x)\text{.}$$ This is called the \textbf{Product Rule}.
\end{dialogue}
\begin{question}
Using the Product Rule, derive the products of the following functions:

$f(x) = \sin(x)+x^2$, $g(x) = 3x^3+x$

$\answer{}$

$f(x) = \cos(x)+4x$, $g(x) = 3x^2+x$

$\answer{}$

$f(x) = x^2$, $g(x) = 3x^3-3x$

$\answer{}$

$f(x) = x^7$, $g(x) = 2x^32$

$\answer{}$

\end{question}

\subsubsection{The Quotient Rule}
\begin{dialogue}
\item[Dylan] Wow! That's gonna save a ton of time with products! Is there anything like it we can do with quotients?
\item[James] There is! It's even called \textbf{the Quotient Rule}!
\item[Julia] I bet it's a pain too though, just like the product rule.
\item[James] Well, why don't you try using your intuition first rather than guessing?
\item[Dylan] Alright, well, I guess I would divide the derivative of the numerator by the derivative of the denominator.
\end{dialogue}
\begin{question}
Consider the functions $f(x) = x^3$ and $g(x) = \cos(x)$.
\[
\graph{x^3, cos(x)}
\]
Use Dylan's guess to find the derivative of $\frac{f(x)}{g(x)}$.

$\answer{3x^2/sin(x)}$

Use the limit definition of the derivative to find the derivative of $\frac{f(x)}{g(x)}$.

$\answer{}$

Was Dylan right?

$\answer{No}$
\end{question}
\begin{dialogue}
\item[Julia] I knew it! It's never that easy!
\item[James] Now calm down Julia, this rule is worse than the last one, but it's much better than going through by hand: $$\frac{d}{dx}*\frac{f(x)}{g(x)} = \frac{f'(x)g(x)-f(x)g'(x)}{g(x)^2}\text{.}$$
\end{dialogue}
\begin{question}
Using the Product Rule, derive the products of the following functions:

$f(x) = \sin(x)+x^2$, $g(x) = 3x^3+x$

$\answer{}$

$f(x) = \cos(x)+4x$, $g(x) = 3x^2+x$

$\answer{}$

$f(x) = x^2$, $g(x) = 3x^3-3x$

$\answer{}$

$f(x) = x^7$, $g(x) = 2x^32$

$\answer{}$
\end{question}
\subsubsection{The Chain Rule}
\begin{dialogue}
\item[James] There's one last rule to learn today; the \textbf{Chain Rule}.
\item[Dylan] That rule sounds pretty cool! When do we use it though? I thought we already covered the functions we need to know...
\item[Julia] Yeah, what else is there?
\item[James] We use the chain rule in composition of functions, like when we have $\sin(2x)$ - $2x$ is a function, and so is $\sin()$!
\item[Julia] And how bad is the rule?
\item[James] This one is a little more tricky - $$\frac{d}{dx}f(g(x)) = f'(g(x))*g'(x)\text{.}$$
\item[Dylan and Julia] That's so gross.
\item[James] Well, let's give it a try and see if you like it more than the limit definition!
\end{dialogue}
\begin{question}
Consider $f(x) = \cos(x)$ and $g(x) = x^3$
\[
\graph{cos(x), x^3}
\]

Using the limit definition of derivative, evaluate the derivative of $f(g(x))$.

$\answer{}$

Now, evaluate the same limit using the chain rule. Was it any better?

$\answer{Yes}$
\end{question}
\begin{question}

Using the Chain Rule, derive the compositions $f(g(x))$ for the following functions:

$f(x) = 3x+x^2$, $g(x) = x^4+7x$

$\answer{}$

$f(x) = \cos(x)$, $g(x) = \sin(x)$

$\answer{}$

$f(x) = x^2-5x$, $g(x) = \sqrt{x+3}$

$\answer{}$

$f(x) = x^7$, $g(x) = \sin(x)-x^3+3$

$\answer{}$
\end{question}
\end{document}

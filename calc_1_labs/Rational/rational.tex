\documentclass{ximera}
\title{Rational Functions with Awful Questions}
\begin{abstract}
\end{abstract}
\begin{document}
\subsubsection{Introduction}
\begin{dialogue}
\item[James] Hey guys, I slept through class yesterday... could you fill me in on what a rational function is?
\item[Julia] See, class didn't make a lot of sense to me because I was thinking, ``Functions can be rational?''
\item[Dylan] They don't mean rational like me or you, Julia! It means \textit{the function can be represented as a fraction where the numerator and denominator are both polynomials}.
\item[Julia and James] Oh!
\item[Dylan] Rational functions are pretty neat, because they can have two different types of discontinuities!
\item[Altogether] LET'S DIVE IN!
\end{dialogue}
\subsubsection{Guided Example}
Consider the function $f(x)=\frac{(x-2)(x+4)}{(x-3)(x+3)(x+4)}$

Graph the function using your favorite CAS system. Depending on the CAS you use, you may need to research how to show discontinuities in a graph. To do this, simply Google "\textit{CAS} show discontinuities", where \textit{CAS} is the name of whatever CAS you are using. At the time this document was written, Desmos did not include discontinuities by default, and thus, a Desmos powered graph has not been provided within this activity.
\begin{question}
Describe the graph. What strange things do you notice?
\begin{freeResponse}
\end{freeResponse}
\end{question}
The "hole" present in the graph is called a \textbf{removable discontinuity}.

The curve which goes vertical is called a \textbf{vertical asymptote}, another type of discontinuity.

\subsubsection{On Your Own}

Find and report the discontinuities in the following functions:

$a(x) = \dfrac{x^2+1}{x-2}$

$b(x) = \dfrac{x^2-5x+7}{x^2-x-6}$

$c(x) = \dfrac{x^2-x}{x}$

$d(x) = \dfrac{x^2-5x+7}{x^3-6x^2+8x-3}$

$f(x) = \dfrac{2x^2+5}{x^2-25}$

$g(x) = \dfrac{x^3-x^2-15x-9}{x+3}$

$h(x) = \dfrac{1}{3x^2-x}$

\begin{question}
How can you tell if a rational function has a vertical asymptote or a removable discontinuity?
\begin{freeResponse}
\end{freeResponse}
How can you find these discontinuities?
\begin{freeResponse}
\end{freeResponse}
\end{question}

\subsubsection{In Summary}
\begin{dialogue}
\item[James] These functions are pretty neat! What were they called again?
\item[Dylan] They're called \textbf{rational functions}, \textit{fractions where the numerator and denominator are both polynomials}!
\item[Julia] So, when exactly does a \textbf{vertical asymptote} occur?
\item[James] I know this one! \textbf{Vertical asymptotes} \textit{occur at points where the denominator of the function will be zero, but the numerator is non-zero}!
\item[Julia] That makes sense! But when do removable discontinuities occur then?
\item[Dylan] \textbf{Removable discontinuities} \textit{occur where the numerator and denominator are both zero}.
\end{dialogue}
\end{document}

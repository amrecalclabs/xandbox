<<<<<<< HEAD
\documentclass{ximera}
=======
\documentclass[handout]{ximera}
\title{Derivative}
\begin{abstract}
\end{abstract}
>>>>>>> 985ffb55d244dd64f9f1c0e4df9b15754ba6021f
\begin{document}
\maketitle
\begin{dialogue}
\item[Julia] Ah, this sucks!
\item[Dylan] What's up?
\item[Julia] I'm supposed to find the slope of a parabola at a point, and I'm not sure how!
\item[Dylan] Well, what if we just make a secant line on the function?
\item[Julia] Secant line? What's that?
\item[Dylan] \textit{A secant line is just a line which connects two points on a function}!
\end{dialogue}

\subsection{Guided Example}

Consider the function $$f(x) = x^2$$
%\begin{image}
%\begin{center}
%\begin{tikzpicture}
%\begin{axis}[
%    axis lines = left,
%    xlabel = $x$,
%    ylabel = {$f(x)$},
%]
%Below the red parabola is defined
%\addplot [
%    domain=-10:10, 
%    samples=100, 
%    color=red,
%]
%{x^2};
%\addlegendentry{$x^2$}
%\end{axis}
%\end{tikzpicture}
%\end{center}
%\end{image}

\begin{question} Find the slope between $x = 2$ and $x = 7$. Does this seem to be a good approximation for the rate of change at $x = 2$? Why or why not? $\answer{9}$ \end{question}
\begin{question}
Dylan thinks we can solve the problem by just picking something closer than 10. What is the slope between $x = 2$ and $x = 3$? $\answer{5}$ 
\end{question}
%Is this a good approximation? Is it better than the last attempt?
\begin{dialogue}
\item[Julia] Dylan, this still isn't a great approximation...
\item[Dylan] Well, I think we need to get even closer. Like, infinitesimally close! But how would we do that....
\item[James] You guys need some help?
\item[Julia and Dylan] James! How do we find the slope of a line at a point?
\item[James] It isn't too tough! Before, you were considering a certain point as your comparison. What if instead, you used the point you want to evaluate at plus something really small? Let's call it $h$.
\end{dialogue}
How can you make the $h$ in $$\frac{f(a+h)-f(a)}{(a+h)-a}$$ become a value closer and closer to zero when we evaluate it?
Using the method you determined in the previous question, approximate the rate of change at the point x=2.
\begin{dialogue}
\item[James] The value at that point is the slope of the tangent line!
\item[Dylan] What's a tangent line?
\item[James] \textit{A tangent line is a line which intersects a differentiable curve at a point where the slope of the curve equals the slope of the line.}. Want to know something really cool?
\item[Julia and Dylan] What James?
\item[James] The function you just discovered is how you determine a function's derivative! Using that process, you can find the rate of change at any point on a function!
\item[Julia and Dylan] Wow! So cool!
\end{dialogue}

\subsection{On Your Own}
Using what you've learned, find the derivative of the following functions at the given point.
\setcounter{problem}{0}
\begin{question}
$g(x) = x^5-5x^4-x^2+2x+1$, $x=2$ $\answer{-82}$
\end{question}
\begin{question}
$h(x) = \frac{1}{x}$, $x=2$ $\answer{-0.25}$
\end{question}

By replacing the $a$ in our formula for the derivative with $x$, we may determine the derivative at any point on the function. Determine the derivative for the following functions.
\begin{question}
$m(x) = x^3$ $\answer{3x^2}$
\end{question}
\begin{question}
$n(x) = 3x+2$ $\answer{3}$
\end{question}


\subsection{In Summary}
\begin{dialogue}
\item[Julia] So why is it called a secant line?
\item[James] It comes from the Latin word secare which means to cut.
\item[Dylan] Ohh, I get it now! Because a secant line is any line that connects two points on a function!
\end{dialogue}
In this lab we've (hopefully) learned the function for finding a derivative using the limit as h approaches 0. We also learned what secant and tangent lines are. For your convenience, the important definitions are listed below. 
\begin{definition}
 A \textbf{secant line} is any line that connects any two points on a curve.
\end{definition}
\begin{definition}
A \textbf{tangent line} is a line which intersects a differentiable curve at a point where the slope of the curve equals the slope of the line.
\end{definition}
\begin{definition}
The \textbf{derivative} $f'(a)$ is defined by the following limit:
$$f'(a)=\displaystyle \lim_{h\rightarrow 0} \frac{f(a+h)-f(a)}{h}$$
\end{definition}
\end{document}
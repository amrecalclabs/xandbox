\documentclass{ximera}
%\usepackage{todonotes}

\newcommand{\todo}{}

\usepackage{esint} % for \oiint
\graphicspath{
{./}
{functionsOfSeveralVariables/}
{normalVectors/}
{lagrangeMultipliers/}
{vectorFields/}
{greensTheorem/}
{shapeOfThingsToCome/}
}


\usepackage{tkz-euclide}
\tikzset{>=stealth} %% cool arrow head
\tikzset{shorten <>/.style={ shorten >=#1, shorten <=#1 } } %% allows shorter vectors

\usetikzlibrary{backgrounds} %% for boxes around graphs
\usetikzlibrary{shapes,positioning}  %% Clouds and stars
\usetikzlibrary{matrix} %% for matrix
\usepgfplotslibrary{polar} %% for polar plots
\usetkzobj{all}
\usepackage[makeroom]{cancel} %% for strike outs
%\usepackage{mathtools} %% for pretty underbrace % Breaks Ximera
\usepackage{multicol}
\usepackage{pgffor} %% required for integral for loops


%% http://tex.stackexchange.com/questions/66490/drawing-a-tikz-arc-specifying-the-center
%% Draws beach ball
\tikzset{pics/carc/.style args={#1:#2:#3}{code={\draw[pic actions] (#1:#3) arc(#1:#2:#3);}}}



\usepackage{array}
\setlength{\extrarowheight}{+.1cm}   
\newdimen\digitwidth
\settowidth\digitwidth{9}
\def\divrule#1#2{
\noalign{\moveright#1\digitwidth
\vbox{\hrule width#2\digitwidth}}}





\newcommand{\RR}{\mathbb R}
\newcommand{\R}{\mathbb R}
\newcommand{\N}{\mathbb N}
\newcommand{\Z}{\mathbb Z}


%\renewcommand{\d}{\,d\!}
\renewcommand{\d}{\mathop{}\!d}
\newcommand{\dd}[2][]{\frac{\d #1}{\d #2}}
\newcommand{\pp}[2][]{\frac{\partial #1}{\partial #2}}
\renewcommand{\l}{\ell}
\newcommand{\ddx}{\frac{d}{\d x}}

\newcommand{\zeroOverZero}{\ensuremath{\boldsymbol{\tfrac{0}{0}}}}
\newcommand{\inftyOverInfty}{\ensuremath{\boldsymbol{\tfrac{\infty}{\infty}}}}
\newcommand{\zeroOverInfty}{\ensuremath{\boldsymbol{\tfrac{0}{\infty}}}}
\newcommand{\zeroTimesInfty}
{\ensuremath{\small\boldsymbol{0\cdot \infty}}}
\newcommand{\inftyMinusInfty}{\ensuremath{\small\boldsymbol{\infty - \infty}}}
\newcommand{\oneToInfty}{\ensuremath{\boldsymbol{1^\infty}}}
\newcommand{\zeroToZero}{\ensuremath{\boldsymbol{0^0}}}
\newcommand{\inftyToZero}{\ensuremath{\boldsymbol{\infty^0}}}



\newcommand{\numOverZero}{\ensuremath{\boldsymbol{\tfrac{\#}{0}}}}
\newcommand{\dfn}{\textbf}
%\newcommand{\unit}{\,\mathrm}
\newcommand{\unit}{\mathop{}\!\mathrm}
\newcommand{\eval}[1]{\bigg[ #1 \bigg]}
\newcommand{\seq}[1]{\left( #1 \right)}
\renewcommand{\epsilon}{\varepsilon}
\renewcommand{\phi}{\varphi}


\renewcommand{\iff}{\Leftrightarrow}

\DeclareMathOperator{\arccot}{arccot}
\DeclareMathOperator{\arcsec}{arcsec}
\DeclareMathOperator{\arccsc}{arccsc}
\DeclareMathOperator{\si}{Si}
\DeclareMathOperator{\proj}{\vec{proj}}
\DeclareMathOperator{\scal}{scal}
\DeclareMathOperator{\sign}{sign}


%% \newcommand{\tightoverset}[2]{% for arrow vec
%%   \mathop{#2}\limits^{\vbox to -.5ex{\kern-0.75ex\hbox{$#1$}\vss}}}
\newcommand{\arrowvec}{\overrightarrow}
%\renewcommand{\vec}[1]{\arrowvec{\mathbf{#1}}}
\renewcommand{\vec}{\mathbf}
\newcommand{\veci}{{\boldsymbol{\hat{\imath}}}}
\newcommand{\vecj}{{\boldsymbol{\hat{\jmath}}}}
\newcommand{\veck}{{\boldsymbol{\hat{k}}}}
\newcommand{\vecl}{\boldsymbol{\l}}
\newcommand{\uvec}[1]{\mathbf{\hat{#1}}}
\newcommand{\utan}{\mathbf{\hat{t}}}
\newcommand{\unormal}{\mathbf{\hat{n}}}
\newcommand{\ubinormal}{\mathbf{\hat{b}}}

\newcommand{\dotp}{\bullet}
\newcommand{\cross}{\boldsymbol\times}
\newcommand{\grad}{\boldsymbol\nabla}
\newcommand{\divergence}{\grad\dotp}
\newcommand{\curl}{\grad\cross}
%\DeclareMathOperator{\divergence}{divergence}
%\DeclareMathOperator{\curl}[1]{\grad\cross #1}
\newcommand{\lto}{\mathop{\longrightarrow\,}\limits}

\renewcommand{\bar}{\overline}

\colorlet{textColor}{black} 
\colorlet{background}{white}
\colorlet{penColor}{blue!50!black} % Color of a curve in a plot
\colorlet{penColor2}{red!50!black}% Color of a curve in a plot
\colorlet{penColor3}{red!50!blue} % Color of a curve in a plot
\colorlet{penColor4}{green!50!black} % Color of a curve in a plot
\colorlet{penColor5}{orange!80!black} % Color of a curve in a plot
\colorlet{penColor6}{yellow!70!black} % Color of a curve in a plot
\colorlet{fill1}{penColor!20} % Color of fill in a plot
\colorlet{fill2}{penColor2!20} % Color of fill in a plot
\colorlet{fillp}{fill1} % Color of positive area
\colorlet{filln}{penColor2!20} % Color of negative area
\colorlet{fill3}{penColor3!20} % Fill
\colorlet{fill4}{penColor4!20} % Fill
\colorlet{fill5}{penColor5!20} % Fill
\colorlet{gridColor}{gray!50} % Color of grid in a plot

\newcommand{\surfaceColor}{violet}
\newcommand{\surfaceColorTwo}{redyellow}
\newcommand{\sliceColor}{greenyellow}




\pgfmathdeclarefunction{gauss}{2}{% gives gaussian
  \pgfmathparse{1/(#2*sqrt(2*pi))*exp(-((x-#1)^2)/(2*#2^2))}%
}


%%%%%%%%%%%%%
%% Vectors
%%%%%%%%%%%%%

%% Simple horiz vectors
\renewcommand{\vector}[1]{\left\langle #1\right\rangle}


%% %% Complex Horiz Vectors with angle brackets
%% \makeatletter
%% \renewcommand{\vector}[2][ , ]{\left\langle%
%%   \def\nextitem{\def\nextitem{#1}}%
%%   \@for \el:=#2\do{\nextitem\el}\right\rangle%
%% }
%% \makeatother

%% %% Vertical Vectors
%% \def\vector#1{\begin{bmatrix}\vecListA#1,,\end{bmatrix}}
%% \def\vecListA#1,{\if,#1,\else #1\cr \expandafter \vecListA \fi}

%%%%%%%%%%%%%
%% End of vectors
%%%%%%%%%%%%%

%\newcommand{\fullwidth}{}
%\newcommand{\normalwidth}{}



%% makes a snazzy t-chart for evaluating functions
%\newenvironment{tchart}{\rowcolors{2}{}{background!90!textColor}\array}{\endarray}

%%This is to help with formatting on future title pages.
\newenvironment{sectionOutcomes}{}{} 



%% Flowchart stuff
%\tikzstyle{startstop} = [rectangle, rounded corners, minimum width=3cm, minimum height=1cm,text centered, draw=black]
%\tikzstyle{question} = [rectangle, minimum width=3cm, minimum height=1cm, text centered, draw=black]
%\tikzstyle{decision} = [trapezium, trapezium left angle=70, trapezium right angle=110, minimum width=3cm, minimum height=1cm, text centered, draw=black]
%\tikzstyle{question} = [rectangle, rounded corners, minimum width=3cm, minimum height=1cm,text centered, draw=black]
%\tikzstyle{process} = [rectangle, minimum width=3cm, minimum height=1cm, text centered, draw=black]
%\tikzstyle{decision} = [trapezium, trapezium left angle=70, trapezium right angle=110, minimum width=3cm, minimum height=1cm, text centered, draw=black]

\title{Derivative}
\begin{abstract}
\end{abstract}
\begin{document}
\maketitle
\begin{dialogue}
\item[Julia] Ah, this sucks!
\item[Dylan] What's up?
\item[Julia] I'm supposed to find the slope of a parabola at a point, and I'm not sure how!
\item[Dylan] Well if we had two points we could make a secant line to approximate it!
\item[Julia] Secant line? What's that?
\item[Dylan] A \textit{secant line} is just a line which connects two points on a function!
\item[Julia] But isn't the \textit{tangent} line one that skims a curve at one point? So the slope of the tangent line is the slope at that point! See?
\[
\graph{f(x)=x^2,g(x)=2(x-1)+1}
\]
\item[Dylan] Well do you know how to find the equation for a line with just one point?
\item[Julia]...
\item[James] Come on guys we can approximate the tangent line using the secant line!
\item[Altogether]Let's dive in!
\end{dialogue}



\section{Guided Example}

Consider the function $$f(x) = x^2$$
\[
\graph{f(x)=x^2,g(x)=2(x-1)+1}
\]
%\begin{image}
%\begin{center}
%\begin{tikzpicture}
%\begin{axis}[
%    axis lines = left,
%    xlabel = $x$,
%    ylabel = {$f(x)$},
%]
%Below the red parabola is defined
%\addplot [
%    domain=-10:10, 
%    samples=100, 
%    color=red,
%]
%{x^2};
%\addlegendentry{$x^2$}
%\end{axis}
%\end{tikzpicture}
%\end{center}
%\end{image}

\begin{question}
Find the slope of the secant line between $x = 2$ and $x = 7$.


$\answer{9}$

Does this seem to be a good approximation for the slope of the tangent line at $x = 2$?

\begin{multipleChoice}
\choice{Yes}
\choice[correct]{No}
\end{multipleChoice}

Dylan thinks we can solve the problem by just picking something closer than 7. Find the slope of the secant line between $x = 2$ and $x = 3$.

$\answer{5}$

Is this a good approximation for the slope of the tangent line at $x=2$?

\begin{multipleChoice}
\choice{Yes}
\choice[correct]{No}
\end{multipleChoice}

Is it better than the last attempt?

$\answer{Yes}$
\end{question}
\begin{dialogue}
\item[Julia] Dylan, this still isn't a great approximation...
\item[Dylan] Well, I think we need to get even closer. Like, infinitesimally close! But how would we do that....
\item[James] You guys need some help?
\item[Julia and Dylan] James! How do we find the slope of a line at a point?
\item[James] It isn't too tough! Before, you were considering a certain point as your comparison. What if instead, you used the point you want to evaluate at plus something really small? Let's call it $h$.
\end{dialogue}
\begin{question}
How can you make the $h$ in $$\frac{f(2+h)-f(2)}{(2+h)-2}$$ approach 0?
\begin{multipleChoice}
\choice[correct]{Use $\lim_{h \to 0}$.}
\choice{Use $\lim_{h \to \infty}$.}
\choice{Divide the fraction by $h$.}
\choice{Pick a function $f(x)$ so that $f(2)$ is 0.}
\end{multipleChoice}

Using the method you determined, approximate the slope of the tangent line at the point x=2.

$\answer{4}$

\end{question}
\begin{dialogue}
\item[James] Want to know something really cool?
\item[Julia and Dylan] What James?
\item[James] The function we just discovered is how you determine a function's derivative! Using that process, you can find the instantaneous rate of change at any point on a function!
\item[Julia and Dylan] Wow! So cool!
\end{dialogue}

\section{On Your Own}
Using what you've learned, find the derivative of the following functions at the given point.
\setcounter{problem}{0}
\begin{question}
$g(x) = x^2+1$, $x=2$

$\answer{4}$

$h(x) = \frac{1}{x}$, $x=2$

$\answer{-0.25}$

$f(x)=3x^2+4x+2$, $x=-1$

$\answer{-2}$

$f(t)=\sqrt{t^2+1}$, $x=3$

$\answer{\frac{3}{\sqrt{10}}}$

$f(x) = x+x^{-1}$,$x=4$

$\answer{3.9375}$

\end{question}

By replacing the point in our formula for the derivative with $x$, we may determine the derivative at any point on the function. Determine the derivative for the following functions.
\begin{question}
$m(x) = x^3$

$\answer{3x^2}$

$n(x) = 3x+2$

$\answer{3}$

$f(x)=4-x^2$

$\answer{-2x}$

$f(x) = 12+7x$

$\answer{7}$

$f(t)=\frac{4}{t+1}$

$\answer{\frac{-4}{(x+1)^2}}$

\end{question}


\section{In Summary}
\begin{dialogue}
\item[Julia] So why is it called a secant line?
\item[James] It comes from the Latin word secare, which means 'to cut'.
\item[Dylan] Ohh, I get it now! Because a secant line is a line that 'cuts' a function!
\end{dialogue}
In this lab we've (hopefully) learned the function for finding a derivative using the limit as $h$ approaches 0. We also learned what secant and tangent lines are. For your convenience, the important definitions are listed below.
\begin{definition}
 A \textbf{secant line} is any line that connects any two points on a curve.
\end{definition}
\begin{definition}
A \textbf{tangent line} is a line which intersects a differentiable curve at a point where the slope of the curve equals the slope of the line.
\end{definition}
\begin{definition}
The \textbf{derivative} $f'(a)$ is defined by the following limit:
$$f'(a)=\displaystyle \lim_{h\rightarrow 0} \frac{f(a+h)-f(a)}{h}$$
\end{definition}
\end{document}
\documentclass{ximera}
\title{Intro to Ximera}
\input{../preamble.tex}
\begin{abstract}
\end{abstract}
\begin{document}
\begin{javascript}
 caseInsensitive = function(a,b) {
    return a.toLowerCase() == b.toLowerCase();
  };
\end{javascript}

\begin{dialogue}
\item[Julia] Hi, I'm Julia. Is anyone sitting here?
\item{James} Nope, just me! I'm James by the way. Let's get started on this lab!
\end{dialogue}

First things first - let's answer an easy multiple choice question! Simply click the correct box and then "Check Your Work"!

\begin{question}
Are you ready to learn how to use Ximera for your Calculus course?

\begin{multipleChoice}
\choice{Never!}
\choice{No!}
\choice[correct]{Heck yeah!}
\choice{No way!}
\end{multipleChoice}
\begin{feedback}[correct]
Well great news for you! That's just what we'll do!
\end{feedback}

\end{question}
\begin{dialogue}
\item[Julia] Ah!
\end{dialogue}

\begin{dialogue}
\item[Dylan] Woah, what's this blank box?
\item[Julia] Looks like we put our answer in it? But how do I know how to format it?
\item[James] Don't worry you two! Ximera is pretty smart, so as long as what you put in is equivalent to the answer Ximera knows, it should work fine! Check it out down here!
\end{dialogue}

\begin{question}
Go ahead and put in $2x^2$ into the following blank:

$\answer{2x^2}$

Now, the answer to this box is $2x^2$ as well, but try $2*(x)*(x)$ or $2x*x$!

$\answer{2x^2}$
\end{question}

\end{document}

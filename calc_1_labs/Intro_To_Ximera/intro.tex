\documentclass{ximera}
\title{Intro to Ximera}
%\usepackage{todonotes}

\newcommand{\todo}{}

\usepackage{esint} % for \oiint
\graphicspath{
{./}
{functionsOfSeveralVariables/}
{normalVectors/}
{lagrangeMultipliers/}
{vectorFields/}
{greensTheorem/}
{shapeOfThingsToCome/}
}


\usepackage{tkz-euclide}
\tikzset{>=stealth} %% cool arrow head
\tikzset{shorten <>/.style={ shorten >=#1, shorten <=#1 } } %% allows shorter vectors

\usetikzlibrary{backgrounds} %% for boxes around graphs
\usetikzlibrary{shapes,positioning}  %% Clouds and stars
\usetikzlibrary{matrix} %% for matrix
\usepgfplotslibrary{polar} %% for polar plots
\usetkzobj{all}
\usepackage[makeroom]{cancel} %% for strike outs
%\usepackage{mathtools} %% for pretty underbrace % Breaks Ximera
\usepackage{multicol}
\usepackage{pgffor} %% required for integral for loops


%% http://tex.stackexchange.com/questions/66490/drawing-a-tikz-arc-specifying-the-center
%% Draws beach ball
\tikzset{pics/carc/.style args={#1:#2:#3}{code={\draw[pic actions] (#1:#3) arc(#1:#2:#3);}}}



\usepackage{array}
\setlength{\extrarowheight}{+.1cm}   
\newdimen\digitwidth
\settowidth\digitwidth{9}
\def\divrule#1#2{
\noalign{\moveright#1\digitwidth
\vbox{\hrule width#2\digitwidth}}}





\newcommand{\RR}{\mathbb R}
\newcommand{\R}{\mathbb R}
\newcommand{\N}{\mathbb N}
\newcommand{\Z}{\mathbb Z}


%\renewcommand{\d}{\,d\!}
\renewcommand{\d}{\mathop{}\!d}
\newcommand{\dd}[2][]{\frac{\d #1}{\d #2}}
\newcommand{\pp}[2][]{\frac{\partial #1}{\partial #2}}
\renewcommand{\l}{\ell}
\newcommand{\ddx}{\frac{d}{\d x}}

\newcommand{\zeroOverZero}{\ensuremath{\boldsymbol{\tfrac{0}{0}}}}
\newcommand{\inftyOverInfty}{\ensuremath{\boldsymbol{\tfrac{\infty}{\infty}}}}
\newcommand{\zeroOverInfty}{\ensuremath{\boldsymbol{\tfrac{0}{\infty}}}}
\newcommand{\zeroTimesInfty}
{\ensuremath{\small\boldsymbol{0\cdot \infty}}}
\newcommand{\inftyMinusInfty}{\ensuremath{\small\boldsymbol{\infty - \infty}}}
\newcommand{\oneToInfty}{\ensuremath{\boldsymbol{1^\infty}}}
\newcommand{\zeroToZero}{\ensuremath{\boldsymbol{0^0}}}
\newcommand{\inftyToZero}{\ensuremath{\boldsymbol{\infty^0}}}



\newcommand{\numOverZero}{\ensuremath{\boldsymbol{\tfrac{\#}{0}}}}
\newcommand{\dfn}{\textbf}
%\newcommand{\unit}{\,\mathrm}
\newcommand{\unit}{\mathop{}\!\mathrm}
\newcommand{\eval}[1]{\bigg[ #1 \bigg]}
\newcommand{\seq}[1]{\left( #1 \right)}
\renewcommand{\epsilon}{\varepsilon}
\renewcommand{\phi}{\varphi}


\renewcommand{\iff}{\Leftrightarrow}

\DeclareMathOperator{\arccot}{arccot}
\DeclareMathOperator{\arcsec}{arcsec}
\DeclareMathOperator{\arccsc}{arccsc}
\DeclareMathOperator{\si}{Si}
\DeclareMathOperator{\proj}{\vec{proj}}
\DeclareMathOperator{\scal}{scal}
\DeclareMathOperator{\sign}{sign}


%% \newcommand{\tightoverset}[2]{% for arrow vec
%%   \mathop{#2}\limits^{\vbox to -.5ex{\kern-0.75ex\hbox{$#1$}\vss}}}
\newcommand{\arrowvec}{\overrightarrow}
%\renewcommand{\vec}[1]{\arrowvec{\mathbf{#1}}}
\renewcommand{\vec}{\mathbf}
\newcommand{\veci}{{\boldsymbol{\hat{\imath}}}}
\newcommand{\vecj}{{\boldsymbol{\hat{\jmath}}}}
\newcommand{\veck}{{\boldsymbol{\hat{k}}}}
\newcommand{\vecl}{\boldsymbol{\l}}
\newcommand{\uvec}[1]{\mathbf{\hat{#1}}}
\newcommand{\utan}{\mathbf{\hat{t}}}
\newcommand{\unormal}{\mathbf{\hat{n}}}
\newcommand{\ubinormal}{\mathbf{\hat{b}}}

\newcommand{\dotp}{\bullet}
\newcommand{\cross}{\boldsymbol\times}
\newcommand{\grad}{\boldsymbol\nabla}
\newcommand{\divergence}{\grad\dotp}
\newcommand{\curl}{\grad\cross}
%\DeclareMathOperator{\divergence}{divergence}
%\DeclareMathOperator{\curl}[1]{\grad\cross #1}
\newcommand{\lto}{\mathop{\longrightarrow\,}\limits}

\renewcommand{\bar}{\overline}

\colorlet{textColor}{black} 
\colorlet{background}{white}
\colorlet{penColor}{blue!50!black} % Color of a curve in a plot
\colorlet{penColor2}{red!50!black}% Color of a curve in a plot
\colorlet{penColor3}{red!50!blue} % Color of a curve in a plot
\colorlet{penColor4}{green!50!black} % Color of a curve in a plot
\colorlet{penColor5}{orange!80!black} % Color of a curve in a plot
\colorlet{penColor6}{yellow!70!black} % Color of a curve in a plot
\colorlet{fill1}{penColor!20} % Color of fill in a plot
\colorlet{fill2}{penColor2!20} % Color of fill in a plot
\colorlet{fillp}{fill1} % Color of positive area
\colorlet{filln}{penColor2!20} % Color of negative area
\colorlet{fill3}{penColor3!20} % Fill
\colorlet{fill4}{penColor4!20} % Fill
\colorlet{fill5}{penColor5!20} % Fill
\colorlet{gridColor}{gray!50} % Color of grid in a plot

\newcommand{\surfaceColor}{violet}
\newcommand{\surfaceColorTwo}{redyellow}
\newcommand{\sliceColor}{greenyellow}




\pgfmathdeclarefunction{gauss}{2}{% gives gaussian
  \pgfmathparse{1/(#2*sqrt(2*pi))*exp(-((x-#1)^2)/(2*#2^2))}%
}


%%%%%%%%%%%%%
%% Vectors
%%%%%%%%%%%%%

%% Simple horiz vectors
\renewcommand{\vector}[1]{\left\langle #1\right\rangle}


%% %% Complex Horiz Vectors with angle brackets
%% \makeatletter
%% \renewcommand{\vector}[2][ , ]{\left\langle%
%%   \def\nextitem{\def\nextitem{#1}}%
%%   \@for \el:=#2\do{\nextitem\el}\right\rangle%
%% }
%% \makeatother

%% %% Vertical Vectors
%% \def\vector#1{\begin{bmatrix}\vecListA#1,,\end{bmatrix}}
%% \def\vecListA#1,{\if,#1,\else #1\cr \expandafter \vecListA \fi}

%%%%%%%%%%%%%
%% End of vectors
%%%%%%%%%%%%%

%\newcommand{\fullwidth}{}
%\newcommand{\normalwidth}{}



%% makes a snazzy t-chart for evaluating functions
%\newenvironment{tchart}{\rowcolors{2}{}{background!90!textColor}\array}{\endarray}

%%This is to help with formatting on future title pages.
\newenvironment{sectionOutcomes}{}{} 



%% Flowchart stuff
%\tikzstyle{startstop} = [rectangle, rounded corners, minimum width=3cm, minimum height=1cm,text centered, draw=black]
%\tikzstyle{question} = [rectangle, minimum width=3cm, minimum height=1cm, text centered, draw=black]
%\tikzstyle{decision} = [trapezium, trapezium left angle=70, trapezium right angle=110, minimum width=3cm, minimum height=1cm, text centered, draw=black]
%\tikzstyle{question} = [rectangle, rounded corners, minimum width=3cm, minimum height=1cm,text centered, draw=black]
%\tikzstyle{process} = [rectangle, minimum width=3cm, minimum height=1cm, text centered, draw=black]
%\tikzstyle{decision} = [trapezium, trapezium left angle=70, trapezium right angle=110, minimum width=3cm, minimum height=1cm, text centered, draw=black]

\begin{abstract}
\end{abstract}
\begin{document}
\begin{javascript}
 caseInsensitive = function(a,b) {
    return a.toLowerCase() == b.toLowerCase();
  };
\end{javascript}

\begin{dialogue}
\item[Julia] Hi, I'm Julia. Is anyone sitting here?
\item[James] Nope, just me! I'm James by the way. Let's get started on this lab!
\end{dialogue}

\section{Multiple Choice and Select All}
First things first - let's answer an easy multiple choice question! Simply click the correct box and then "Check Your Work"!

\begin{question}
Are you ready to learn how to use Ximera for your Calculus course?

\begin{multipleChoice}
\choice{Never!}
\choice{No!}
\choice[correct]{Heck yeah!}
\choice{No way!}
\end{multipleChoice}
\begin{feedback}[correct]
Well great news for you! That's just what we'll do!
\end{feedback}

\end{question}
\begin{dialogue}
\item[Dylan] Ah! What was that?
\item[James] Quit yelling in here! That was a feedback box, they usually give you a little more information on the question you answered.
\item[Dylan] Oh, alright. Well, I'm Dylan! It's a pleasure.
\item[Julia] I'm Julia, and this is James.
\end{dialogue}

Let's look at another type of question here: select all. These allow you to pick multiple boxes before checking your answer, and you need to get all of them to get the right answer! These choices will not always be made clear in these labs, so if you think you see two or more right answers, click away!

\begin{question}
Who have we met in this lab?
\begin{selectAll}
\choice{Jim}
\choice{Jeff}
\choice[correct]{Julia}
\choice{Jennifer}
\choice[correct]{James}
\choice{Dillon}
\choice[correct]{Dylan}
\choice{Don}
\end{selectAll}
\end{question}

\section{Fill in the Blanks!}
\begin{dialogue}
\item[Dylan] Woah, what's this blank box?
\item[Julia] Looks like we put our answer in it? But how do I know how to format it?
\item[James] Don't worry you two! Ximera is pretty smart, so as long as what you put in is equivalent to the answer Ximera knows, it should work fine! Check it out down here!
\end{dialogue}

\begin{question}
Go ahead and put in $2x^2$ into the following blank, using \^ \; to raise $x$ to the power of two:

$\answer{2x^2}$

Now, the answer to this box is $2x^2$ as well, but try $2*(x)*(x)$ or $2x*x$!

$\answer{2x^2}$
\begin{feedback}[correct]
Look! It all works the same! Isn't Ximera great?
\end{feedback}
\end{question}

\begin{dialogue}
\item[Dylan] Well that's cool and all, but what if I need a square root?
\item[James] That's easy!
\end{dialogue}

There are two ways to enter a square root in Ximera; sqrt() and raising to the one-half power.

\begin{question}
Using what we learned in the last example, use \^ \; to input $\sqrt{2}$.

$\answer{\sqrt{2}}$

Now, use sqrt(2) to input it here!

$\answer{\sqrt{2}}$
\begin{feedback}[correct]
Notice that Ximera gives you what it thinks you're inputting as you fill in the box! If you keep getting the wrong answer but think you're right, make sure to see if Ximera is interpretting your input correctly!
\end{feedback}
\end{question}

\section{Hints}
\begin{dialogue}
\item[Julia] Ximera is cool, but I'm a little worried. What if I get stuck and I'm doing it outside of class? I can't exactly ask the professor then!
\item[James] That is true Julia, but the people who made this thought of just that! When a problem can be tough or confusing, they sometimes drop you a \textbf{hint}. Look down below, and click the show hint button to see what they can do!
\end{dialogue}

Let's put some tough questions down, and use hints to answer them!
\begin{question}
\begin{hint}
It's one of the characters we've seen so far, and the only one who doesn't have a J in their name!
\end{hint}
Who wrote this lab?
$\answer[validator = caseInsensitive]{Dylan}$
\end{question}
\begin{question}
\begin{hint}
This was three years before 2020.
\end{hint}
What year was this lab written?
$\answer{2017}$
\end{question}

Sometimes, a single question block can have multiple hints - if you're stuck, and there's a hint box, it's always worth clicking it again to see if another hint will appear!

\begin{question}
\begin{hint}
I don't think you need a hint here.
\end{hint}
This question is easy, just click yes!
\begin{multipleChoice}
\choice{No}
\choice{I refuse}
\choice[correct]{yes!}
\choice{Yes}
\end{multipleChoice}

\begin{hint}
My favorite number is $\sqrt{4}$.
\end{hint}
What is my favorite number?
$\answer{2}$
\end{question}

\section{Desmos}

\end{document}

\documentclass{ximera}
\usepackage{tikz}
\begin{document}
\subsection{Differentiation Rules}
\begin{dialogue}
\item[Julia] Hmm...I don't think differentiation rules, it takes so long and I hate using that long limit definition!
\item[Dylan] No no Julia, it's differentiation \textit{rules}!
\item[Julia] Ohhhh, that makes more sense!

\end{dialogue}
\subsubsection{The Power Rule}
\begin{dialogue}
\item[Julia] I hate how long it takes to differentiate powers!
\item[Dylan] Yeah, it takes forever! I feel like there was some sort of pattern to it though. I couldn't figure anything out though.
\item[James] Sounds like you guys need my help again?
\item[Julia and Dylan] Help us James!
\item[James] There \textit{is} a pattern! Check out this table I made!
\begin{center}
\begin{tabular}{c|c}
$f(x)$ & $f'(x)$ \\
\hline
$x^2$ & $2x^1$ \\
$x^3$ & $3x^2$ \\
$x^4$ & $4x^3$ 
\end{tabular}
\end{center}
\end{dialogue}
\begin{question}
What pattern do you notice in James' table? Generalize this pattern in terms of $x^n$
\begin{freeResponse}
This is the model solution %You don’t actually need anything in between the begin and end line.
\end{freeResponse}
\end{question}
\begin{question}
$ \frac{\partial}{\partial x} x^n) =  \answer{n*x^n-1}$
\end{question}

\begin{question}
Using the limit definition of a derivative, compute the derivative for $x^5$, did you get the same answer as you would using the pattern you generalized above?
\begin{freeResponse}
This is the model solution %You don’t actually need anything in between the begin and end line.
\end{freeResponse}
\end{question}


\subsubsection{The Constant Rule}
\begin{dialogue}
\item[Dylan] Wow! That's neat!
\item[Julia] I wish we could use rules like this all over the place though, it would really save me time.
\item[James] There are plenty of places with rules like this! Why don't we look at a function like $y = 3$?
\end{dialogue}

Consider $y = c$, where $c$ is some arbitrary constant.
\begin{question}
Derive this function using the limit definition. What does your answer mean?
\begin{freeResponse}
This is the model solution %You don’t actually need anything in between the begin and end line.
\end{freeResponse}
\end{question}

\begin{question}
Using what you found in the previous problem, compute the following derivatives:
\begin{enumerate}
\item{$f(x)=2$}
\item{$f(x)=100$}
\item{$f(x)=0$}
\end{enumerate}
\begin{freeResponse}
This is the model solution %You don’t actually need anything in between the begin and end line.
\end{freeResponse}
\begin{freeResponse}
This is the model solution %You don’t actually need anything in between the begin and end line.
\end{freeResponse}
\begin{freeResponse}
This is the model solution %You don’t actually need anything in between the begin and end line.
\end{freeResponse}
\end{question}

\subsubsection{The Constant Multiple Rule}
\begin{dialogue}
\item[Julia] James! Show us more! These things are going to save me so much time on my homework!
\item[James] Alright alright, calm down Julia. We can look at a function like $y = 3x$ next.
\end{dialogue}

Consider $y = kx$, where $k$ is some arbitrary constant.
\begin{question}
\item{Derive this function using the limit definition. What does your answer mean?}
\begin{freeResponse}
This is the model solution %You don’t actually need anything in between the begin and end line.
\end{freeResponse}
\end{question}

\begin{question}
\item{Using what you found in the previous problem, compute the following derivatives:}
\item{$f(x)=4x$}
\item{$f(x)=10x$}
\item{$f(x)=\frac{1}{5}x$}
\begin{freeResponse}
This is the model solution %You don’t actually need anything in between the begin and end line.
\end{freeResponse}
\end{question}

\subsubsection{The Sum and Difference Rules}
\begin{dialogue}
\item[Dylan] Wow, this stuff is awesome! Is there any way to put it all together? Like, is there an easy way to tell what the derivative of $f(x) = 3x+4$ is?
\item[James] There is Dylan! 
\end{dialogue}
Consider the differentiable functions $f(x)$ and $g(x)$. We will define a function $j(x) = f(x) + g(x)$.


\begin{enumerate}
\item{Take the derivative of $j(x)$ using the limit definition. What does your answer mean? \textit{Hint: In $j(x+h)$, the $(x+h)$ will replace $x$ in the component functions as well.}}
\item{Using what you found in the previous problem, compute the following derivatives:
\begin{enumerate}
\item{$f(x) = 3x^2 - 5x + 2$, $g(x) = x^2 + 3x$}
\item{$f(x) = x^2 - 4x + 2$, $g(x) = -4x^2 + 3$}
\item{$f(x) = 5x^3 + 3x$, $g(x) = 2x^2 - 13x$}
\end{enumerate}
}
\item{Julia wonders if a similar rule exists for $j(x) = f(x)-g(x)$. Using the limit definition of derivative, determine if there is a pattern. Then, if there is a rule, use it to solve the 1a, 1b, and 1c. If there is not, do them using the limit definition.}
\end{enumerate}

\subsubsection{In Summary}
We've covered a lot of differentiation rules in this lab, to help you out we've made the following table.
\begin{center}
{\renewcommand{\arraystretch}{3}
\begin{tabular}{| l | p{7.5cm} |}
    \hline
    Power Rule & $\displaystyle \dfrac{d}{dx}(x^n)=n*x^{(n-1)}$,where $n$ is any real number besides 0. \\
    \hline
    Constant Rule & $\displaystyle\dfrac{d}{dx}(c) = 0$ \\
    \hline
    Constant Multiple Rule & $\displaystyle\dfrac{d}{dx}(cf(x))=c\dfrac{d}{dx}f(x)$ \\
    \hline
    Sum Rule & $\displaystyle\dfrac{d}{dx}(f(x)+g(x))=\dfrac{d}{dx}f(x)+\dfrac{d}{dx}g(x)$ \\
    \hline
    Difference Rule & $\displaystyle\dfrac{d}{dx}(f(x)-g(x))=\dfrac{d}{dx}f(x)-\dfrac{d}{dx}g(x)$ \\
    \hline
\end{tabular}}
\end{center}
\pagebreak
\end{document}
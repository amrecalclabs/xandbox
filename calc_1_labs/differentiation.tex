\documentclass{ximera}
\usepackage{tikz}
\title{Differentiation Rules!}
\begin{abstract}
\end{abstract}
\begin{document}
\maketitle
\begin{dialogue}
\item[Julia] Hmm...I don't think differentiation rules, it takes so long and I hate using that long limit definition!
\item[Dylan] No no Julia, it's differentiation \textit{rules}!
\item[Julia] Ohhhh, that makes more sense!

\end{dialogue}
\section{The Power Rule}
\begin{dialogue}
\item[Julia] I hate how long it takes to differentiate powers!
\item[Dylan] Yeah, it takes forever! I feel like there was some sort of pattern to it though. I couldn't figure anything out though.
\item[James] Sounds like you guys need my help again?
\item[Julia and Dylan] Help us James!
\item[James] There \textit{is} a pattern! Check out this table I made!
\end{dialogue}
\begin{center}
\begin{tabular}{c|c}
$f(x)$ & $\frac{d}{dx}f(x)$ \\
\hline
$x^2$ & $2x^1$ \\
$x^3$ & $3x^2$ \\
$x^4$ & $4x^3$
\end{tabular}
\end{center}
\begin{question}
What pattern do you notice in James' table? Generalize this pattern in terms of $x^n$.

\begin{multipleChoice}
\choice[correct]{$n \cdot x^{n-1}$}
\choice{$n-1 \cdot x^{n-1}$}
\choice{$n \cdot x^n$}
\choice{$n-1 \cdot x^n$}
\end{multipleChoice}
\end{question}

\begin{question}
Using the limit definition of a derivative, compute the derivative for $x^5$

$\frac{d}{dx} x^5 =  \answer{5x^4}$

Notice that your answer fits the same pattern as before!
\end{question}

\begin{question}
Use the power rule to differentiate the following functions.

$f(x) = x^{10}$ \hspace{11mm} $\frac{d}{dx}f(x) = \answer{10x^9}$

$f(x) = 3x^2$ \hspace{10mm} $\frac{d}{dx}f(x) = \answer{6x}$

\begin{hint}
The value $\frac{1}{x}$ can be represented by $x^{-1}$.
\end{hint}
$f(x) = \frac{5}{x}$ \hspace{12mm} $\frac{d}{dx}f(x) = \answer{-5x^{-2}}$
\end{question}

\section{The Constant Rule}
\begin{dialogue}
\item[Dylan] Wow! That's neat!
\item[Julia] I wish we could use rules like this all over the place though, it would really save me time.
\item[James] There are plenty of places with rules like this! Why don't we look at a function like $y = 3$?
\end{dialogue}

Consider $y = c$, where $c$ is some arbitrary constant.
\begin{question}
Derive this function using the limit definition.

$ \frac{d}{dx} c =  \answer{0}$

What does your answer mean?
\begin{freeResponse}
\end{freeResponse}
\end{question}

\begin{question}
Using what you found in the previous problem, compute the following derivatives:

$f(x)=2$ \hspace{14mm} $\frac{d}{dx} f(x) =  \answer{0}$

$f(x)=100$ \hspace{10mm} $\frac{d}{dx} f(x) =  \answer{0}$

$f(x)=0$ \hspace{13mm} $\frac{d}{dx} f(x) =  \answer{0}$

\end{question}

\section{The Constant Multiple Rule}
\begin{dialogue}
\item[Julia] James! Show us more! These things are going to save me so much time on my homework!
\item[James] Alright alright, calm down Julia. We can look at a function like $y = 3x$ next.
\end{dialogue}

Consider $y = k \cdot x$, where $k$ is some arbitrary constant.
\begin{question}
$\frac{d}{dx} (k \cdot x) =  \answer{k}$

\item{What does your answer mean?}

\begin{freeResponse}

\end{freeResponse}
\end{question}

\begin{question}
\item{Using what you found in the previous problem, compute the following derivatives:}

$f(x) = 4x$ \hspace{12mm} $\frac{d}{dx} f(x) =  \answer{4}$

$f(x) = 10x$ \hspace{10mm} $\frac{d}{dx} f(x) =  \answer{10}$

$f(x) = \frac{1}{5}x$ \hspace{11mm} $\frac{d}{dx} f(x) =  \answer{\frac{1}{5}}$
\end{question}

\section{The Sum and Difference Rules}
\begin{dialogue}
\item[Dylan] Wow, this stuff is awesome! Is there any way to put it all together? Like, is there an easy way to tell what the derivative of $f(x) = 3x+4$ is?
\item[James] There is Dylan!
\end{dialogue}

\begin{question}
Consider the differentiable functions $f(x)$ and $g(x)$. We will define a function $j(x) = f(x) + g(x)$.

\begin{hint}
In $j(x+h)$, the $(x+h)$ will replace $x$ in the component functions as well.
\end{hint}
Take the derivative of $j(x)$ using the limit definition.

$j'(x) =  \answer{\frac{\j(x+h)-j(x)}{h}}$

What does your answer mean?

\begin{freeResponse}
\end{freeResponse}
\end{question}
\begin{question}
Using what you found in the previous problem, compute the following derivatives:

$f(x) = 3x^2 - 5x + 2$, $g(x) = x^2 + 3x$ \hspace{11mm} $\frac{d}{dx}j(x) =  \answer{8x-2}$

$f(x) = x^2 - 4x + 2$, $g(x) = -4x^2 + 3$ \hspace{10mm} $\frac{d}{dx}j(x) = \answer{-6x-4}$

$f(x) = 5x^3 + 3x$, $g(x) = 2x^2 - 13x$ \hspace{13mm} $\frac{d}{dx}j(x) =  \answer{15x^2+4x-10}$
\end{question}
\begin{question}
Julia wonders if a similar rule exists for $m(x) = f(x)-g(x)$. Using the limit definition of derivative, determine if there is a pattern. Then, if there is a rule, use it to solve the 1a, 1b, and 1c. If there is not, do them using the limit definition.

$f(x) = 3x^2 - 5x + 2$, $g(x) = x^2 + 3x$ \hspace{11mm} $\frac{d}{dx}m(x) =  \answer{4x-8}$

$f(x) = x^2 - 4x + 2$, $g(x) = -4x^2 + 3$ \hspace{10mm} $\frac{d}{dx}m(x) =  \answer{-10x-4}$

$f(x) = 5x^3 + 3x$, $g(x) = 2x^2 - 13x$ \hspace{13mm} $\frac{d}{dx}m(x) =  \answer{15x^2-4x+16}$
\end{question}
\section{In Summary}
We've covered a lot of differentiation rules in this lab, to help you out we've made the following table.
\begin{center}
\renewcommand{\arraystretch}{3}
\begin{tabular}{| l | p{7.5cm} |}
    \hline
    Power Rule & $\dfrac{d}{dx}x^n=$$\answer{n}$\begin{multipleChoice}
    \choice[correct]{-}
    \choice{$\div$}
    \choice{$\cdot$}
    \choice{+}
    \end{multipleChoice} $\answer{x^{n-1}}$,where $n$ is any real number. \\
    \hline
    Constant Rule & $\dfrac{d}{dx}c =$ $\answer{0}$ \\
    \hline
    Constant Multiple Rule & $\dfrac{d}{dx}(c \cdot f(x))=$ $\answer{c}$ \begin{multipleChoice}
    \choice[correct]{-}
    \choice{$\div$}
    \choice{$\cdot$}
    \choice{+}
    \end{multipleChoice} $\dfrac{d}{dx}$ $\answer{f(x)}$ \\
    \hline
    Sum Rule & $\dfrac{d}{dx}(f(x)+g(x))=$ $\dfrac{d}{dx}$ $\answer{f(x)}$ \begin{multipleChoice}
    \choice[correct]{-}
    \choice{$\div$}
    \choice{$\cdot$}
    \choice{+}
    \end{multipleChoice} $\dfrac{d}{dx}$ $\answer{g(x)}$ \\
    \hline
    Difference Rule & $\dfrac{d}{dx}(f(x)-g(x))=$ $\dfrac{d}{dx}$ $\answer{f(x)}$
    \begin{multipleChoice}
    \choice[correct]{-}
    \choice{$\div$}
    \choice{$\cdot$}
    \choice{+}
    \end{multipleChoice}
    $\dfrac{d}{dx}$ $\answer{g(x)}$ \\
    \hline
\end{tabular}
\end{center}

\pagebreak
\end{document}
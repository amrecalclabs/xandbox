\documentclass{ximera}
\title{Newton's Methods QUESTIONS}
\begin{abstract}
\end{abstract}
\begin{document}
\maketitle
\section{Introduction}
\begin{dialogue}
\item[Dylan] I'm so tired of having to solve roots by hand. It's a real drag.
\item[Julia] Yeah, some of these roots are rough. I wish there was a better way!
\item[James] There's always a better way!
\item[Dylan and Julia] Show us!!!
\item[James] Maybe you've heard of Sir Isaac Newton? He got tired of solving roots too, and made a whole method to approximate them!
\item[Dylan] Wow! I'm just like him except worse in every way!
\end{dialogue}
Newton's Method is a system of approximating roots of polynomials by using tangent lines from an initial estimate. While this method is extremely accurate when used properly, it is possible to have a very inaccurate estimate when used improperly.
\section{Guided Example}
In the following figure we have an initial guess $x_{0}$, then we have the blue tangent line with respect to the point $x_{0}$

\begin{question}
What is the slope, in general, for the tangent line of $y=f(x)$ at $x_{0}$?

$\answer{f'(x_0)}$

What is the equation of the tangent line for the point $(x_{0},f(x_{0})$? Please answer in slope-intercept form.

$\answer{y = f'(x)x_0+b}$

How would you use the tangent line you found above to estimate the value of $x_{1}$?

\begin{freeResponse}
\end{freeResponse}

\end{question}
\section{On Your Own}
\begin{question}
Consider the function $f(x) = x^2-1$.
\[
\graph{x^2-1}
\]

Find the tangent line at an initial estimate of $x_0=3$.

$\answer{}$

Plot the tangent line and function on the same axes. Does the x-intercept of the tangent line seem more or less accurate than your initial estimate?

\begin{multipleChoice}
\choice[correct]{More Accurate}
\choice{Less Accurate}
\end{multipleChoice}

What is the x-intercept of the tangent line?

$\answer{I don't know}$

Continue this process until the x-intercepts change by less than .0001 on each interval.
\end{question}
Consider the function $g(x) = x^3-4x^2-1$.
\[
\graph{x^3-4x^2-1}
\]
Using the same method as before, estimate a root of $g(x)$ using your own initial guess.

Explain why the function has only one solution with the help of a graph.
\[
\graph{}
\]
\begin{freeResponse}
\end{freeResponse}
Using $g(x)$ from the previous problem, use an initial guess of 2. After 5 iterations, what result do you get?
$\answer{A really bad one}$

Why is it important to use caution with Newton's method?
\begin{freeResponse}
\end{freeResponse}
\section{In Summary}
\begin{dialogue}
\item[Julia] Wow! Newton's Method is awesome!
\item[Dylan] Yeah, it's way more accurate than just guessing! If you're too far off on that initial guess though...
\item[James] Things can go downhill quickly. While Newton's Method can be handy, it's important to remember how important an accurate initial estimate is!
\item[Dylan and Julia] Thanks James!
\end{dialogue}
\end{document}
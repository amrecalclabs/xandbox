\documentclass{ximera}
%\usepackage{todonotes}

\newcommand{\todo}{}

\usepackage{esint} % for \oiint
\graphicspath{
{./}
{functionsOfSeveralVariables/}
{normalVectors/}
{lagrangeMultipliers/}
{vectorFields/}
{greensTheorem/}
{shapeOfThingsToCome/}
}


\usepackage{tkz-euclide}
\tikzset{>=stealth} %% cool arrow head
\tikzset{shorten <>/.style={ shorten >=#1, shorten <=#1 } } %% allows shorter vectors

\usetikzlibrary{backgrounds} %% for boxes around graphs
\usetikzlibrary{shapes,positioning}  %% Clouds and stars
\usetikzlibrary{matrix} %% for matrix
\usepgfplotslibrary{polar} %% for polar plots
\usetkzobj{all}
\usepackage[makeroom]{cancel} %% for strike outs
%\usepackage{mathtools} %% for pretty underbrace % Breaks Ximera
\usepackage{multicol}
\usepackage{pgffor} %% required for integral for loops


%% http://tex.stackexchange.com/questions/66490/drawing-a-tikz-arc-specifying-the-center
%% Draws beach ball
\tikzset{pics/carc/.style args={#1:#2:#3}{code={\draw[pic actions] (#1:#3) arc(#1:#2:#3);}}}



\usepackage{array}
\setlength{\extrarowheight}{+.1cm}   
\newdimen\digitwidth
\settowidth\digitwidth{9}
\def\divrule#1#2{
\noalign{\moveright#1\digitwidth
\vbox{\hrule width#2\digitwidth}}}





\newcommand{\RR}{\mathbb R}
\newcommand{\R}{\mathbb R}
\newcommand{\N}{\mathbb N}
\newcommand{\Z}{\mathbb Z}


%\renewcommand{\d}{\,d\!}
\renewcommand{\d}{\mathop{}\!d}
\newcommand{\dd}[2][]{\frac{\d #1}{\d #2}}
\newcommand{\pp}[2][]{\frac{\partial #1}{\partial #2}}
\renewcommand{\l}{\ell}
\newcommand{\ddx}{\frac{d}{\d x}}

\newcommand{\zeroOverZero}{\ensuremath{\boldsymbol{\tfrac{0}{0}}}}
\newcommand{\inftyOverInfty}{\ensuremath{\boldsymbol{\tfrac{\infty}{\infty}}}}
\newcommand{\zeroOverInfty}{\ensuremath{\boldsymbol{\tfrac{0}{\infty}}}}
\newcommand{\zeroTimesInfty}
{\ensuremath{\small\boldsymbol{0\cdot \infty}}}
\newcommand{\inftyMinusInfty}{\ensuremath{\small\boldsymbol{\infty - \infty}}}
\newcommand{\oneToInfty}{\ensuremath{\boldsymbol{1^\infty}}}
\newcommand{\zeroToZero}{\ensuremath{\boldsymbol{0^0}}}
\newcommand{\inftyToZero}{\ensuremath{\boldsymbol{\infty^0}}}



\newcommand{\numOverZero}{\ensuremath{\boldsymbol{\tfrac{\#}{0}}}}
\newcommand{\dfn}{\textbf}
%\newcommand{\unit}{\,\mathrm}
\newcommand{\unit}{\mathop{}\!\mathrm}
\newcommand{\eval}[1]{\bigg[ #1 \bigg]}
\newcommand{\seq}[1]{\left( #1 \right)}
\renewcommand{\epsilon}{\varepsilon}
\renewcommand{\phi}{\varphi}


\renewcommand{\iff}{\Leftrightarrow}

\DeclareMathOperator{\arccot}{arccot}
\DeclareMathOperator{\arcsec}{arcsec}
\DeclareMathOperator{\arccsc}{arccsc}
\DeclareMathOperator{\si}{Si}
\DeclareMathOperator{\proj}{\vec{proj}}
\DeclareMathOperator{\scal}{scal}
\DeclareMathOperator{\sign}{sign}


%% \newcommand{\tightoverset}[2]{% for arrow vec
%%   \mathop{#2}\limits^{\vbox to -.5ex{\kern-0.75ex\hbox{$#1$}\vss}}}
\newcommand{\arrowvec}{\overrightarrow}
%\renewcommand{\vec}[1]{\arrowvec{\mathbf{#1}}}
\renewcommand{\vec}{\mathbf}
\newcommand{\veci}{{\boldsymbol{\hat{\imath}}}}
\newcommand{\vecj}{{\boldsymbol{\hat{\jmath}}}}
\newcommand{\veck}{{\boldsymbol{\hat{k}}}}
\newcommand{\vecl}{\boldsymbol{\l}}
\newcommand{\uvec}[1]{\mathbf{\hat{#1}}}
\newcommand{\utan}{\mathbf{\hat{t}}}
\newcommand{\unormal}{\mathbf{\hat{n}}}
\newcommand{\ubinormal}{\mathbf{\hat{b}}}

\newcommand{\dotp}{\bullet}
\newcommand{\cross}{\boldsymbol\times}
\newcommand{\grad}{\boldsymbol\nabla}
\newcommand{\divergence}{\grad\dotp}
\newcommand{\curl}{\grad\cross}
%\DeclareMathOperator{\divergence}{divergence}
%\DeclareMathOperator{\curl}[1]{\grad\cross #1}
\newcommand{\lto}{\mathop{\longrightarrow\,}\limits}

\renewcommand{\bar}{\overline}

\colorlet{textColor}{black} 
\colorlet{background}{white}
\colorlet{penColor}{blue!50!black} % Color of a curve in a plot
\colorlet{penColor2}{red!50!black}% Color of a curve in a plot
\colorlet{penColor3}{red!50!blue} % Color of a curve in a plot
\colorlet{penColor4}{green!50!black} % Color of a curve in a plot
\colorlet{penColor5}{orange!80!black} % Color of a curve in a plot
\colorlet{penColor6}{yellow!70!black} % Color of a curve in a plot
\colorlet{fill1}{penColor!20} % Color of fill in a plot
\colorlet{fill2}{penColor2!20} % Color of fill in a plot
\colorlet{fillp}{fill1} % Color of positive area
\colorlet{filln}{penColor2!20} % Color of negative area
\colorlet{fill3}{penColor3!20} % Fill
\colorlet{fill4}{penColor4!20} % Fill
\colorlet{fill5}{penColor5!20} % Fill
\colorlet{gridColor}{gray!50} % Color of grid in a plot

\newcommand{\surfaceColor}{violet}
\newcommand{\surfaceColorTwo}{redyellow}
\newcommand{\sliceColor}{greenyellow}




\pgfmathdeclarefunction{gauss}{2}{% gives gaussian
  \pgfmathparse{1/(#2*sqrt(2*pi))*exp(-((x-#1)^2)/(2*#2^2))}%
}


%%%%%%%%%%%%%
%% Vectors
%%%%%%%%%%%%%

%% Simple horiz vectors
\renewcommand{\vector}[1]{\left\langle #1\right\rangle}


%% %% Complex Horiz Vectors with angle brackets
%% \makeatletter
%% \renewcommand{\vector}[2][ , ]{\left\langle%
%%   \def\nextitem{\def\nextitem{#1}}%
%%   \@for \el:=#2\do{\nextitem\el}\right\rangle%
%% }
%% \makeatother

%% %% Vertical Vectors
%% \def\vector#1{\begin{bmatrix}\vecListA#1,,\end{bmatrix}}
%% \def\vecListA#1,{\if,#1,\else #1\cr \expandafter \vecListA \fi}

%%%%%%%%%%%%%
%% End of vectors
%%%%%%%%%%%%%

%\newcommand{\fullwidth}{}
%\newcommand{\normalwidth}{}



%% makes a snazzy t-chart for evaluating functions
%\newenvironment{tchart}{\rowcolors{2}{}{background!90!textColor}\array}{\endarray}

%%This is to help with formatting on future title pages.
\newenvironment{sectionOutcomes}{}{} 



%% Flowchart stuff
%\tikzstyle{startstop} = [rectangle, rounded corners, minimum width=3cm, minimum height=1cm,text centered, draw=black]
%\tikzstyle{question} = [rectangle, minimum width=3cm, minimum height=1cm, text centered, draw=black]
%\tikzstyle{decision} = [trapezium, trapezium left angle=70, trapezium right angle=110, minimum width=3cm, minimum height=1cm, text centered, draw=black]
%\tikzstyle{question} = [rectangle, rounded corners, minimum width=3cm, minimum height=1cm,text centered, draw=black]
%\tikzstyle{process} = [rectangle, minimum width=3cm, minimum height=1cm, text centered, draw=black]
%\tikzstyle{decision} = [trapezium, trapezium left angle=70, trapezium right angle=110, minimum width=3cm, minimum height=1cm, text centered, draw=black]

\usepackage{tikz}
\title{Differentiation Rules!}
\begin{abstract}
\end{abstract}
\begin{document}
\maketitle
\begin{dialogue}
\item[Julia] Hmm...I don't think differentiation rules, it takes so long and I hate using that long limit definition!
\item[Dylan] No no Julia, it's differentiation \textit{rules}!
\item[Julia] Ohhhh, that makes more sense!

\end{dialogue}
\section{The Power Rule}
\begin{dialogue}
\item[Julia] I hate how long it takes to differentiate powers!
\item[Dylan] Yeah, it takes forever! I feel like there was some sort of pattern to it, but I couldn't figure anything out.
\item[James] Sounds like you guys need my help again?
\item[Julia and Dylan] Help us James!
\item[James] There \textit{is} a pattern! Check out this table I made!
\begin{center}
\begin{tabular}{c|c}
$f(x)$ & $\ddx f(x)$ \\
\hline
$x^2$ & $2x^1$ \\
$x^3$ & $3x^2$ \\
$x^4$ & $4x^3$
\end{tabular}
\end{center}
\end{dialogue}
\begin{question}
What pattern do you notice in James' table? Generalize this pattern in terms of $x^n$.
\begin{multipleChoice}
\choice[correct]{$n \cdot x^{n-1}$}
\choice{$n-1 \cdot x^{n-1}$}
\choice{$n \cdot x^n$}
\choice{$n-1 \cdot x^n$}
\end{multipleChoice}
\begin{feedback}[correct]
Congrats! You've found what's known as the \dfn{Power Rule}!
\end{feedback}
\end{question}

\begin{definition}
  The \textbf{derivative} of $f(x)$ at $a$ is defined by the following limit:
  \[
  \eval{\frac{d}{dx} f(x)}_{x=a} = \lim_{h\to 0} \frac{f(a+h) - f(a)}{h}.
  \]
\end{definition}

\begin{question}
Using the limit definition of a derivative, compute the derivative for $x^3$.

$\frac{d}{dx} x^3 =  \answer{3x^2}$
\begin{feedback}[correct]
Notice that your answer fits the same pattern as before!
\end{feedback}
\end{question}

\begin{question}
Use the power rule to differentiate the following functions.

$f(x) = x^{10}$ \hspace{11mm} $\ddx f(x) = \answer{10x^9}$

$f(x) = 3x^2$ \hspace{10mm} $\ddx f(x) = \answer{6x}$

\begin{hint}
The value $\frac{1}{x}$ can be represented by $x^{-1}$.
\end{hint}
$f(x) = \frac{5}{x}$ \hspace{12mm} $\frac{d}{dx}f(x) = \answer{-5x^{-2}}$
\end{question}

\section{The Constant Rule}
\begin{dialogue}
\item[Dylan] Wow! That's neat!
\item[Julia] I wish we could use rules like this all over the place though, it would really save me time.
\item[James] There are plenty of places with rules like this! Why don't we look at a function like $y = 3$?
\end{dialogue}

Consider $y = c$, where $c$ is some arbitrary constant.
\begin{question}
\item{Differentiate this function using the limit definition.}

$ \frac{d}{dx} c =  \answer{0}$

What can you generalize about the derivative of $y=c$ based on this?
\begin{multipleChoice}
\choice{$\frac{d}{dx} c = 2c$}
\choice[correct]{$\frac{d}{dx} c = 0$}
\choice{$\frac{d}{dx} c = x$}
\choice{$\frac{d}{dx} c = \frac{c}{2}$}
\end{multipleChoice}
\begin{feedback}[correct]
Congrats! You've found what's known as the \dfn{Constant Rule}!
\end{feedback}
\end{question}

\begin{question}
Using what you found in the previous question, compute the following derivatives:

$f(x)=2$ \hspace{14mm} $\frac{d}{dx} f(x) =  \answer{0}$

$f(x)=100$ \hspace{10mm} $\frac{d}{dx} f(x) =  \answer{0}$

$f(x)=0$ \hspace{13mm} $\frac{d}{dx} f(x) =  \answer{0}$

\end{question}

\section{The Constant Multiple Rule}
\begin{dialogue}
\item[Julia] James! Show us more! These things are going to save me so much time on my homework!
\item[James] Alright alright, calm down Julia. We can look at a function like $y = 3x$ next.
\end{dialogue}

Consider $y = kx$, where $k$ is some arbitrary constant.
\begin{question}
Differentiate this function using the limit definition:
$\frac{d}{dx} (kx) = \answer{k}$

What can you generalize about the derivative of $y = kx$ based on this?

\begin{multipleChoice}
\choice{$\frac{d}{dx} kx = 2k$}
\choice{$\frac{d}{dx} kx = kx^2$}
\choice[correct]{$\frac{d}{dx} kx = k$}
\choice{$\frac{d}{dx} kx = x$}
\end{multipleChoice}
\begin{feedback}[correct]
Congrats! You've found what's known as the \dfn{Constant Multiple Rule}!
\end{feedback}
\end{question}

\begin{question}
\item{Using what you found in the previous problem, compute the following derivatives:}

$f(x) = 4x$ \hspace{12mm} $\frac{d}{dx} f(x) =  \answer{4}$

$f(x) = 10x$ \hspace{10mm} $\frac{d}{dx} f(x) =  \answer{10}$

$f(x) = \frac{1}{5}x$ \hspace{11mm} $\frac{d}{dx} f(x) =  \answer{\frac{1}{5}}$
\end{question}

\section{The Sum and Difference Rules}
\begin{dialogue}
\item[Dylan] Wow, this stuff is awesome! Is there any way to put it all together? Like, is there an easy way to tell what the derivative of $f(x) = 3x+4$ is?
\item[James] There is Dylan!
\end{dialogue}

\begin{question}
Consider the differentiable functions $f(x)$ and $g(x)$. We will define a function $j(x) = f(x) + g(x)$.

\begin{hint}
In $j(x+h)$, the $(x+h)$ will replace $x$ in $f(x)$ and $g(x)$ as well.
\end{hint}
Take the derivative of $j(x)$ using the limit definition.

$j'(x) =  \answer{\frac{\j(x+h)-j(x)}{h}}$

What can you generalize based on this?

\begin{multipleChoice}
\choice{$f'(x) + g(x) = f'(x) - g'(x)$}
\choice{$f'(x) + g(x) = g'(x) + f'(x)$}
\choice{$f'(x) + g(x) = f(x)\cdot g'(x) - g(x) \cdot f'(x)$}
\choice[correct]{$f'(x) + g(x) = f'(x) + g'(x)$}
\end{multipleChoice}
\begin{feedback}[correct]
Congrats! You've found what's known as the \dfn{Sum Rule}!
\end{feedback}
\end{question}
\begin{question}
Using what you found in the previous problem, compute the following derivatives:

$f(x) = 3x^2 - 5x + 2$, $g(x) = x^2 + 3x$ \hspace{11mm} $\frac{d}{dx}j(x) =  \answer{8x-2}$

$f(x) = x^2 - 4x + 2$, $g(x) = -4x^2 + 3$ \hspace{10mm} $\frac{d}{dx}j(x) = \answer{-6x-4}$

$f(x) = 5x^3 + 3x$, $g(x) = 2x^2 - 13x$ \hspace{13mm} $\frac{d}{dx}j(x) =  \answer{15x^2+4x-10}$
\end{question}
\begin{question}
Julia wonders if a similar rule exists for $m(x) = f(x)-g(x)$. Using the limit definition of derivative, determine if there is a pattern. Then, if there is a rule, use it to solve the 1a, 1b, and 1c. If there is not, do them using the limit definition.

$f(x) = 3x^2 - 5x + 2$, $\;g(x) = x^2 + 3x$ \hspace{11mm} $\frac{d}{dx}m(x) =  \answer{4x-8}$

$f(x) = x^2 - 4x + 2$, $\;g(x) = -4x^2 + 3$ \hspace{10mm} $\frac{d}{dx}m(x) =  \answer{10x-4}$

$f(x) = 5x^3 + 3x$, $\;g(x) = 2x^2 - 13x$ \hspace{13mm} $\frac{d}{dx}m(x) =  \answer{15x^2-4x+16}$
\end{question}
\section{In Summary}
We've covered a lot of differentiation rules in this lab, to help you out we've made the following table.
\begin{question}
\[
\begin{array}{l|l}
\hline
  \text{Power Rule}&\begin{prompt}\ddx x^n=\answer{n}\answer{*}\answer{x^n-1}\end{prompt}\\
  \text{Constant Rule} & \ddx c=\answer{0}\\
  \text{Constant Multiple Rule} & \begin{prompt}\ddx c\cdot f(x)=\answer{c}\answer{*}\ddx \answer{f(x)}\end{prompt}\\
  \text{Sum Rule} & \begin{prompt}\ddx (f(x)+g(x))=\ddx \answer{f(x)}\answer{+}\ddx \answer{g(x)}\end{prompt}\\
  \text{Difference Rule} & \begin{prompt}\ddx (f(x)-g(x))=\ddx \answer{f(x)}\answer{-}\ddx \answer{g(x)}\end{prompt}
\end{array}
\]
\end{question}

\pagebreak
\end{document}
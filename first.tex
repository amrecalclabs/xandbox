\documentclass{ximera}

\title{This is the sample!}
\author{A Ximera Author}

\begin{abstract}
  This is a place to get started.
\end{abstract}

\begin{document}
\maketitle

Here's a sample question.

\begin{problem}
\begin{multipleChoice}
\choice{Incorrect}
\choice{Not this one!}
\choice[correct]{Click here?}
\choice{Not me!}
\end{multipleChoice}
\end{problem}

\begin{problem}
  \begin{multipleChoice}
  \choice[correct]{YES!}
  \choice{No, no no.}
  \end{multipleChoice}
\end{problem}

Blah blah.  I'm typing TeX code here.

\begin{problem}
   You can test that $x + x = \answer{2x}$ or that $x \cdot x = \answer{x^2}$.

  I can do calculations like 
  \[
    \sqrt{\answer{4}} = 2
  \]
  and
  \[
    \frac{\answer{1}}{2} = 0.5
  \]
\end{problem}

\begin{problem}
  Set up the definite integral
    \[
      \int_{\answer{0}}^{\answer{1}} f(x) \, dx
    \]
    to evaluate from $0$ to $1$.

  \begin{question}
    What if $f(x) = \sin x$?  Then,
    \[
       \int_0^1 \sin x \, dx = \answer{-\cos(1) + \cos(0)}.
    \]
  \end{question}
\end{problem}

\begin{problem}
   Expressed to six decimal places,  $\pi \approx \answer[tolerance=0.0000005]{3.141592653}$
   \begin{hint}
   Did you round?
   \end{hint}
\end{problem}

\begin{problem}
   The tolerance 17 means $3421 \approx \answer[tolerance=17]{3421}$
\end{problem}

\end{document}

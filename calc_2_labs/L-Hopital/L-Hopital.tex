\documentclass{ximera}
\title{L'H\^{o}pital's Rule Practice}
\begin{abstract}
\end{abstract}

\usepackage{../Calc2_preamble}

\newtheorem*{LRule}{L'H\^{o}pital's Rule}
\newcommand{\LHop}{L'H\^{o}pital}

\begin{document}
\maketitle

\section{Introduction}
\begin{dialogue}
    \item[Julia] Something something frustrated something
    \item[Dylan] Math pun or related joke on the topic
    \item[James] Deus Ex James to the rescue...
\end{dialogue}

Recall that \LHop's Rule says the following:

\begin{theorem}
    Suppose we have that the limit $\dlim_{x \to a} \df{f(x)}{g(x)}$ has indeterminate form $\frac{0}{0}$ or $\frac{\infty}{\infty}$. Then
    \[
        \dlim_{x \to a} \df{f(x)}{g(x)} = \dlim_{x \to a} \df{f'(x)}{g'(x)},
    \]
provided $g'(x) \neq 0$ around $a$ and that $\dlim_{x \to a} \df{f'(x)}{g'(x)}$ exists or is infinite.
\end{theorem}


Note that
    \begin{itemize}
        \item This rule is valid if you replace $a$ with $\pm \infty$.
        \item This rule is valid for one-sided limits as well.
        \item The \emph{indeterminate form} $\frac{0}{0}$ means $\dlim_{x \to a} f(x) = 0$ and $\dlim_{x \to a} g(x) = 0$.
        \item The \emph{indeterminate form} $\frac{\infty}{\infty}$ means $\dlim_{x \to a} f(x) = \infty$ and $\dlim_{x \to a} g(x) = \infty$.
        \item You can apply L'H\^{o}pital's Rule more than once!! As long as the hypothesis regarding the indeterminate form is satisfied, you can apply the rule again and again.
        \item \emph{Never do the quotient rule!!!} It's the derivative of the top over the derivative of the bottom - NOT the derivative of the quotient!!
    \end{itemize} 


\section*{Indeterminate Forms}

The following are all \emph{indeterminate forms} for limits that you might encounter:
\[
    \frac{0}{0}, \frac{\infty}{\infty}, 0 \cdot \infty, \infty - \infty, 0^0, 1^{\infty}
\]

\begin{itemize}
\item For the form $0 \cdot \infty$, use algebra to make the limit be in $\frac{0}{0}$ or $\frac{\infty}{\infty}$ form.

\item For the form $\infty - \infty$, you'll generally need to combine fractions to get it into $\frac{0}{0}$ or $\frac{\infty}{\infty}$ form.

\item For either of $0^0$ or $1^{\infty}$, you need to use logarithms.
%Set the expression in the limit equal to $y$ and examine instead $\ln(y)$. Compute the limit of $\ln(y)$ to get an answer, then

\end{itemize}

\section{Problems for Practice}

In your group, try to use L'H\^{o}pital's Rule to determine the limit. Make sure you check first if L'H\^{o}pital's Rule applies. If it doesn't, you might have to change the limit with some algebra to be able to use L'H\^{o}pital's Rule.

    \begin{problem}
     $\dlim_{x \to \infty} \df{x}{e^x} = \answer{0}$
    \end{problem}
    \begin{problem}
     $\dlim_{x \to 0} \df{\tan(x)}{x} = \answer{1}$
    \end{problem}
    \begin{problem}
     $\dlim_{x \to 0} \df{2e^x + x - 2}{\sin(x)} = \answer{3}$
    \end{problem}
    \begin{problem}
     $\dlim_{x \to \pi} \df{\sin^2(x)}{1+\cos(x)} = \answer{2}$
    \end{problem}
    \begin{problem}
    \begin{hint}
    $x\ln(x) = \frac{\ln(x)}{\frac{1}{x}}$
    \end{hint}
    $\dlim_{x \to 0^+} x \ln(x) = \answer{0}$
    \end{problem}
    \begin{problem}
    $\dlim_{x \to -\infty} x\sin\left(\frac{1}{x}\right) = \answer{1}$
    \end{problem}
    \begin{problem}
    \begin{hint}
    Make quotients with Trig!
    \end{hint}
    $\dlim_{x \to \frac{\pi}{2}} \sec(x) - \tan(x) = \answer{0}$
    \end{problem}

    \begin{problem}
    To compute $\dlim_{x \to 0^+} x^x$, we use logs. Let $y=x^x$. Then $\ln(y) = x \ln(x)$. Now compute:
    \[
        \dlim_{x \to 0^+} \ln(y)
    \]


    $$\answer{0}$$


    Using continuity, we have $\dlim_{x \to 0^+} \ln(y) = \ln\left(\dlim_{x \to 0^+} y\right)$. Use your answer above and your knowledge of exponential functions to determine now $\dlim_{x \to 0^+} x^x$.

    $$\answer{0}$$

    \end{problem}

    \begin{problem}
Use the same tricks as the last problem to compute:
        \[
            \dlim_{x \to 1} \left(1+\ln(x)\right)^{\frac{1}{x-1}}
        \]

    $$\answer{e}$$
     \end{problem}





\end{document}

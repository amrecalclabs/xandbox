\documentclass{ximera}
\title{Logarithms}
\begin{abstract}
\end{abstract}
\begin{document}
\maketitle
\section{Introduction}
\begin{dialogue}
\item[Dylan] $\log_b(x)$? Why are they talking about trees on this paper?
\item[Julia] Well, that doesn't seem quite right... it probably isn't talking about forests or anything. Is it?
\item[James] Come on you guys, the lecture $just$ went over this! It's the inverse exponential function!
\end{dialogue}

\section{Examining Log Rules}
\begin{dialogue}
\item[Dylan] Alright, so it isn't about trees, and maybe I wasn't paying attention during the lecture. So, what do I need to know before I do the lab?
\item[Julia] Well, at least you're admitting it! I think I remember us going over a few rules for logarithms, but I can't quite seem to remember how they went...
\item[James] Let's do a refresher then!
\end{dialogue}
For the following multiple choice questions, you'll be given the left hand side of the equation. Match it up with the right hand side!
\begin{question}
log$_b\left(\dfrac{x}{y}\right)$

\begin{multipleChoice}
\choice{$\log_b(y)- \log_b(x)$}
\choice{$\dfrac{\log_b(x)}{\log_b(y)}$}
\choice[correct]{$\log_b(x) - \log_b(y)$}
\choice{$\dfrac{\log_b(y)}{\log_b(x)}$}
\end{multipleChoice}

log$_b(x*y)$
\begin{multipleChoice}
\choice{$\log_b(y) \div \log_b(x)$}
\choice{$\log_b(x) \cdot \log_b(y)$}
\choice{$\log_b(y) \cdot \log_b(x)$}
\choice[correct]{$\log_b(x) + \log_b(y)$}
\end{multipleChoice}

log$_b(x^y)$
\begin{multipleChoice}
\choice[correct]{$y$ $\cdot$ log$_b(x)$}
\choice{$x \cdot \log_b(y)$}
\choice{$\log_b(x^y)$}
\choice{$\log_b(y^x)$}
\end{multipleChoice}
\end{question}

\section{Beyond the Basics}
\begin{dialogue}
\item[James] Now that we've gotten past the basic stuff, let's talk about the meaty stuff - calculus with logs!
\item[Dylan] I'm not going to like this, am I?
\item[Julia] Sometimes you're way too into this James...
\end{dialogue}
\begin{example}
Lets work together to determine the value of $\dfrac{d}{dx}\ln(x)$
\begin{explanation}
First, lets think about a number that might make it easier for us to determine the value.
\begin{multipleChoice}
\choice{$x^2$}
\choice{$x$}
\choice[correct]{$e^x$}
\choice{$\pi$}
\end{multipleChoice}
\begin{feedback}
We want $e^x$ because it is its own derivative - that will make our differentiation more than a bit easier!
\end{feedback}

Now, lets consider a general equation $y = \ln(e^y)$.

What is the derivative of the left hand side?
$\dfrac{d}{dy}y = \answer{1}$

On the right hand side, we apply the chain rule to see $$\dfrac{d}{dy} \ln(e^y) = \dfrac{dx}{dy} \cdot \dfrac{d}{dx}\ln(x) \text{, where } x = e^y$$

Now, because we know that $\dfrac{d}{dy}e^y = e^y$, we can change our $\dfrac{dx}{dy}\cdot\dfrac{d}{dx}\ln(x) = 1$ to what?

\begin{multipleChoice}
\choice{$x$}
\choice{$\dfrac{d}{dz} = x$}
\choice{$\dfrac{1}{\dfrac{d}{dx}}$}
\choice[correct]{$x*\dfrac{d}{dx}\ln(x)$}
\end{multipleChoice}
\begin{feedback}
And thus, we see that we have $\dfrac{d}{dx}\ln(x) = \dfrac{1}{x}$. So in general, the derivative of $\ln(x)$ is $\dfrac{1}{x}$!
\end{feedback}
\end{explanation}
\end{example}

\begin{dialogue}
\item[Dylan] Well, that was quite a bit James!
\item[James] Its good to know!
\item[Julia] Do you think you could give us a little practice James? I wanna be sure I understand how to use it to get an A on this coming exam!
\item[James] It would be my pleasure!
\end{dialogue}

\section{Practice with Logarithmic Differentiation}
Take the derivative of the following functions, without using any technology (except to enter your answer!).
\begin{problem}
$\ln(\ln(x)^3)=\answer{\dfrac{3}{x\cdot \ln(x)}}$
\end{problem}

\begin{problem}
$\ln(\cot(x))=\answer{-\csc(x)\cdot\sec(x)}$
\end{problem}

Now, use your knowledge of Sage (and if necessary, a quick glance back at the Intro to Sage lab) to evaluate the following derivatives at the indicated point.

\begin{onlineOnly}
\begin{sageCell}

\end{sageCell}
\end{onlineOnly}

\begin{problem}
$\ln(\sin(\cos(\ln(x))))$ at $x=10$

$\answer{-1/10\cdot \sin\cdot\ln(10)\cdot\cot(\cos(\ln(10)))}$
\end{problem}

\begin{problem}
$16^{\ln(\csc(x)^2)}$ at $x = 1$

$\answer{-3\cdot\ln(16)\cdot\cot(1)\cdot 16^{3\ln(\csc(1))}}$
\end{problem}

Finally, we'll look at the general form of the derivative of a logarithmic function with any base $b$. Here's a hint to get you started:

Use your knowledge of $\dfrac{d}{dx} \ln(x)$ and the change of base formula, $\log_b(x) = \dfrac{\ln(x)}{\ln(b)}$ to find the derivative for $\log_b(x)$.
\begin{problem}
What is $\dfrac{d}{dx}\log_b(x)$?
$\answer{\frac{1}{x\cdot\ln(b)}}$
\end{problem}
\end{document}
\documentclass{ximera}
\title{Exponentials}
\begin{abstract}
\end{abstract}
\begin{document}
\maketitle
\section{Introduction}
\begin{dialogue}
\item[Dylan] Hey Julia, can you help me with this derivative?
\item[Julia] Sure, which one is it? They've been pretty easy so far.
\item[Dylan] I can't figure out $2^x$.
\item[Julia] Oh, I just did $x \cdot 2^{x-1}$.
\end{dialogue}
Let's look at what Julia did and see if it makes sense.

\begin{question}

Below are $2^x$ and $x \cdot 2^{x-1}$ graphed on the same set of axes.
\[
\graph{2^x,x*2^{x-1}}
\]


Does it seem like $x \cdot 2^{x-1}$ is really the graph of the derivative?

\begin{multipleChoice}
\choice{Yes}
\choice[correct]{No}
\end{multipleChoice}

\end{question}
\section{Guided Example}
\begin{dialogue}
\item[Dylan] Maybe we could go to office hours and get some help with this? I really don't understand what I'm supposed to do.
\item[Julia] What if we called James? He always knows what to do!
\item[James] Y'all need help?
\item[Julia and Dylan] James! How did you get here?
\item[Julia] I didn't even call you yet...
\item[James] Don't worry about it guys. Anyway, let's look at the limit definition of the derivative for this one. $$\dfrac{d}{dx}(2^x)=\lim_{h \to 0} \dfrac{2^{x+h}-2^x}{h}$$
\end{dialogue}

\begin{question}

Manipulate the definition James gave to factor out $2^x$ from the limit.
\begin{multipleChoice}
\choice{$2\cdot x \dfrac{2^h-1}{h}$}
\choice{$2 \cdot x \dfrac{2^h-2}{h}$}
\choice{$2^x \cdot \dfrac{2^h-2}{h}$}
\choice[correct]{$2^x \cdot \dfrac{2^h-1}{h}$}
\end{multipleChoice}

Convince yourself that this limit exists. You may zoom in on the graph at the $y$-axis, or use progressively smaller values of $h$ to prove this to yourself.
\[
\graph{}
\]

\begin{onlineOnly}
\begin{sageCell}

\end{sageCell}
\end{onlineOnly}

Notice that the derivative is a constant times $f(x)$. Create a graph with $y$ equal to the constant you found, and on the same axes plot $\ln \left(x\right)$. Where is the intersection?

\begin{multipleChoice}
\choice{$0.712$}
\choice[correct]{$0.693$}
\choice{$0.684$}
\choice{$0.671$}
\end{multipleChoice}

Because the intersection is there, what is your constant equivalent to?

\begin{multipleChoice}
\choice{$0.5^2$}
\choice{$\log_{10}(2)$}
\choice{$\dfrac{1}{2}$}
\choice[correct]{$\ln\left(2\right)$}
\end{multipleChoice}

Repeat this process for $3^x$ and see if you obtain similar results.

Where is the intersection located?

\begin{multipleChoice}
\choice[correct]{$1.0986$}
\choice{$1.0934$}
\choice{$1.0094$}
\choice{$1.0731$}
\end{multipleChoice}

Because the intersection is there, what is your constant equivalent to?

\begin{multipleChoice}
\choice{$0.5^3$}
\choice{$\log_{10}(3)$}
\choice{$\dfrac{1}{3}$}
\choice[correct]{$\ln\left(3\right)$}
\end{multipleChoice}
\end{question}

\section{On Your Own}
\begin{question}
Based on your results from the previous section, what is $\dfrac{d}{dx}(a^x)$ for any $a>0$?
\begin{multipleChoice}
\choice{$a^x$}
\choice{$\ln\left(h\right) \cdot \dfrac{a^x-1}{h}$}
\choice{$\ln\left(a\right) \cdot \dfrac{a^h-1}{h}$}
\choice[correct]{$a^x \cdot \dfrac{a^h-1}{h}$}
\end{multipleChoice}

\item Now, we would like to see a value for which $\displaystyle \lim_{h \to 0} \dfrac{a^h-1}{h} = 1$. What would this mean $\dfrac{d}{dx}(a^x)$ would equal?
\begin{multipleChoice}
\choice[correct]{$a^x$}
\choice{$\ln\left(a\right)$}
\choice{$\ln\left(x\right)$}
\choice{$x^a$}
\end{multipleChoice}


Using Sage, numerically evaluate the limit at $a = 2$ and $a = 3$. How do they relate to the value we're looking for?

\begin{onlineOnly}
\begin{sageCell}

\end{sageCell}
\end{onlineOnly}

\begin{multipleChoice}
\choice{Both 2 and 3 are too large.}
\choice{Both 2 and 3 are too small.}
\choice[correct]{The value is between 2 and 3.}
\end{multipleChoice}

Using what you just noticed, use Sage, along with trial and error, to attempt to find the $a$ for which the limit will be one.

\begin{onlineOnly}
\begin{sageCell}

\end{sageCell}
\end{onlineOnly}

What value do you find?
\begin{multipleChoice}
\choice{2.3}
\choice{2.1}
\choice{2.69}
\choice[correct]{2.71}
\choice{3.14}
\choice{1.8}
\end{multipleChoice}

\begin{dialogue}
\item[Dylan] Hey, this looks familiar...
\item[Julia] I swear I've seen that before!
\item[James] That's $e$! Euler discovered this constant, and its unique properties have made it a \textit{natural} choice for a logarithmic base, leading to a plethora of names for it! $e$ itself is also known as Euler's number and the Naperian base, and when used as a logarithmic base, it is shown as $\ln \left(x\right)$ and known as the natural log!
\end{dialogue}

To confirm this is the case use Sage to evaluate $\dfrac{d}{dx}(e^x)$.

\begin{onlineOnly}
\begin{sageCell}

\end{sageCell}
\end{onlineOnly}

What result do you get?
$\answer{e^x}$
\end{question}

\begin{dialogue}
\item[Julia] Well, I guess we found something pretty cool!
\item[Dylan] I guess it's cool that we found something another mathematician did, but what's the point? Like, that's neat that it is its own derivative, but is there any other reason to know it?
\item[James] $e$ is extremely common in mathematics Dylan! Right now, the money in your savings account is being affected by it!
\item[Dylan] What?! What are you talking about?!
\end{dialogue}

\section{A Simple Application}
When money is put into a savings account with a growth rate of $r$, it grows by a factor of $1 + r$ at the end of each year. This means that, at the end of each year, your funds will be $$P_n = P_{n-1}+P_{n-1} \cdot r= P_{n-1}(1+r)\text{,}$$ where $P_{0}$ is your initial balance, or principal, and $P_{n}$ is your balance after $n$ years.

Now, imagine if, for whatever reason, your bank wanted to apply half that rate to your account, twice per year, i.e., at the end of the year your balance would be $$P_n = P_{n-1}\left(1+\dfrac{r}{2}\right)\left(1+\dfrac{r}{2}\right) = P_{n-1}\left(1+\dfrac{r}{2}\right)^2\text{.}$$

In general, the change in balance when compounded $n$ times per year is $$P_n = P_{n-1}\left(1+\dfrac{r}{n}\right)^n\text{.}$$

\begin{question}

For all $r > 0$, what is the relationship between $\left(1+\dfrac{r}{2}\right)^2$ and $(1 + r)$?

\begin{multipleChoice}
\choice{$(1+r) \leq \left(1+\dfrac{r}{2}\right)^2$}
\choice{$\left(1+\dfrac{r}{2}\right)^2 \leq (1+r)$}
\choice{$(1+r) = \left(1+\dfrac{r}{2}\right)^2$}
\choice{$\left(1+\dfrac{r}{2}\right)^2 < (1+r)$}
\end{multipleChoice}

Determine the factor your balance grows by for the following intervals.
\begin{itemize}
\item Quarterly
\begin{multipleChoice}
\choice[correct]{$\left(1+\dfrac{r}{4}\right)^4$}
\choice{$\left(1+\dfrac{r}{48}\right)^48$}
\choice{$\left(1+\dfrac{r}{3}\right)^3$}
\choice{$\left(1+\dfrac{r}{25}\right)^25$}
\end{multipleChoice}
\item Monthly
\begin{multipleChoice}
\choice{$\left(1+\dfrac{r}{38}\right)^38$}
\choice{$\left(1+\dfrac{r}{48}\right)^48$}
\choice[correct]{$\left(1+\dfrac{r}{12}\right)^12$}
\choice{$\left(1+\dfrac{r}{35}\right)^35$}
\end{multipleChoice}
\item Daily
\begin{multipleChoice}
\choice{$\left(1+\dfrac{r}{36}\right)^36$}
\choice[correct]{$\left(1+\dfrac{r}{365}\right)^365$}
\choice{$\left(1+\dfrac{r}{380}\right)^380$}
\choice{$\left(1+\dfrac{r}{24}\right)^24$}
\end{multipleChoice}
\item Hourly
\begin{multipleChoice}
\choice[correct]{$\left(1+\dfrac{r}{8760}\right)^8760$}
\choice{$\left(1+\dfrac{r}{525600}\right)^525600$}
\choice{$\left(1+\dfrac{r}{365}\right)^365$}
\choice{$\left(1+\dfrac{r}{8640}\right)^8640$}
\end{multipleChoice}
\end{itemize}
As the number of compoundings gets larger and large, the multiplication factor becomes $$\lim_{n \to \infty}\left(1 + \dfrac{r}{n}\right)^n \text{.}$$

Substitute $r = 1$ into the factor, and evaluate using your CAS. What is your result?
\begin{onlineOnly}
\begin{sageCell}

\end{sageCell}
\end{onlineOnly}

\begin{multipleChoice}
\choice{$\infty$}
\choice{$1$}
\choice{$\pi$}
\choice[correct]{$e$}
\end{multipleChoice}

Evaluate the limit for the following values of $r$:
\begin{itemize}
\item $r = 0.3$
\begin{multipleChoice}
\choice{$1.42$}
\choice[correct]{$e^{0.3}$}
\choice{$\dfrac{e}{3}$}
\choice{$1.33$}
\end{multipleChoice}
\item $r = 0.1$
\begin{multipleChoice}
\choice{$\dfrac{e}{10}$}
\choice{$1$}
\choice{$1.12$}
\choice[correct]{$e^{0.1}$}
\end{multipleChoice}
\item $r = 0.7$
\begin{multipleChoice}
\choice[correct]{$e^{0.7}$}
\choice{$\dfrac{e}{7}$}
\choice{$1.023$}
\choice{$e^7$}
\end{multipleChoice}
\item $r$, the general case
\begin{multipleChoice}
\choice{$\dfrac{1}{10}\cdot r$}
\choice{$\dfrac{e}{r}$}
\choice[correct]{$e^r$}
\choice{$r$}
\end{multipleChoice}
\end{itemize}
\end{question}
\end{document}
\documentclass{ximera}
\begin{document}
\title{James Document for self learning}
\section{Application of the Chain Rule}

\begin{dialogue}
\item[Julia] I love class, but I keep wondering why I'm even learning this stuff. I'm not a math major.
\item[Dylan] It isn't like we're ever going to use this stuff in our lives. It's all just theoretical.
\item[James] Hold on guys! Actually, we use derivatives all the time - it is a way of measuring change after all.
\item[Dylan] No way man, I can forget all this after class. Give me one time I'd use a derivative other than class.
\item[James] I'll give you three!
\end{dialogue}

\begin{abstract}
  Here we compute derivatives of compositions of functions
\end{abstract}
\maketitle

\subsubsection{The Great Molasses Flood}
On January 15, 1919, a molasses storage tank in Boston burst, sending molasses rushing down the streets at 35 miles per hour. \footnote{https://www.scientificamerican.com/article/molasses-flood-physics-science/}

Let's pretend something similar happens in Wooster! Imagine you're on the street, walking by our newly installed molasses tank when it begins to burst. Unfortunately, you're by Born, and the molasses is rushing down the hill towards you with its position modeled by $$\frac{1}{5}t^2+t \text{,}$$ Your position can be modeled by $$3t+45 \text{.}$$ In both cases, $t$ is measured in seconds, with each equation reporting a position in meters.

\begin{enumerate}
\item{What is your speed at any point? What about the speed of the molasses?}
\item{What is your acceleration? The acceleration of the molasses?}
\item{How quickly is the speed of the molasses changing after one minute?}
\item{If you want to survive the flood, you'll need to get off the street and into a tall, sturdy building. Born is only 10 seconds away, but there is a group of people trying to get in, meaning once you are there, it will take 20 seconds to reach the inside of the building. Bissman has very little foot traffic, but you'll take exactly 20 seconds to get there and inside. Which building should you go to?}
\end{enumerate}


\subsection{Marginal Profit}
A company that makes peanut butter has a profit of $$P(x)=-0.0027x^3+0.05x^2+18x-125$$, where $x$ is the units produced. One unit of peanut butter contains 10,000 jars and the profit is in thousands of dollars.
\begin{enumerate}
\item{Compute the marginal profit $P'(x)$, explain what is meant by marginal profit.} 
\item{Use the marginal profit function to approximate the increase in profit when production is increased from 20 units to 21 units.}
\item{Use the marginal profit function to approximate the increase in profit when production is increased from 65 to 66 units.}
\item{Graph the marginal profit function, how would you change production based on this graph if the company was currently producing 20 units? What about 65 units?}
\end{enumerate}

\subsection{Dorm Room Froyo}
You've opened up a Froyo franchise in your dorm room! It's a little cramped, but people are hearing about it and enjoying your generous pricing and the convenient location. We can model how many people hear about your franchise with the equation $$p(t) = \frac{1}{1000}t^2 + 2t \text{.}$$ We can also model the profit of your location with the equation $$t(p) = (p^2 + \frac{10}{3}p - 7000)^{\frac{1}{4}} \text{,}$$ where $t$ is time in days, and $p$ is the population of people who are willing to come to your franchise each day.

\begin{enumerate}
\item{If you start with no customers, how many days will it take you to start profiting?}
\item{Using the Chain Rule, how will your profit be changing 35 days from now? Explain exactly what your answer means.}
\end{enumerate}
\pagebreak
\end{document}




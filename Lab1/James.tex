\documentclass{ximera}
\begin{document}
\title{James Document for learning}
\section{Application of the Chain Rule}

\begin{abstract}
Here we compute derivatives of compositions of functions
\end{abstract}

\begin{dialogue}
\item[Julia] I love class, but I keep wondering why I'm even learning this stuff. I'm not a math major.
\item[Dylan] It isn't like we're ever going to use this stuff in our lives. It's all just theoretical.
\item[James] Hold on guys! Actually, we use derivatives all the time - it is a way of measuring change after all.
\item[Dylan] No way man, I can forget all this after class. Give me one time I'd use a derivative other than class.
\item[James] I'll give you three!
\end{dialogue}


So far we have seen how to compute the derivative of a function built up from other functions by addition, subtraction, multiplication and division. There is another very important way that we combine functions: composition. The \textit{chain rule} allows us to deal with this case. Consider
\[
h(x) = \sin(1+2x).
\] 
While there are several different ways to differentiate this function, if we let $f(x) = \sin(x)$ and $g(x) = 1+2x$, then we can express $h(x) = f(g(x))$. The question is, can we compute the derivative of a composition of functions using the derivatives of the constituents $f(x)$ and $g(x)$? To do so, we need the \textit{chain rule}.


\begin{theorem}[Chain Rule]\index{chain rule}\index{derivative rules!chain}
If $f$ and $g$ are differentiable, then
\[
\dfrac{d}{dx} f(g(x)) = f'(g(x))g'(x)
\]
\end{theorem}


It will take a bit of practice to make the use of the chain rule come
naturally, it is more complicated than the earlier differentiation
rules we have seen. Let's return to our motivating example.

\begin{example}
Compute:
\[
\dfrac{d}{dx} \sin(1+2x)
\]

\begin{explanation}
Set $f(x) = \sin(x)$ and $g(x) = 1+2x$, now
\[
f'(x) = \answer[given]{\cos(x)} \qquad\text{and}\qquad g'(x) = \answer[given]{2}.
\]
Hence
\begin{align*}
\dfrac{d}{dx} \sin(1+2x) &= \dfrac{d}{dx} f(g(x)) \\
 &=f'(g(x))g'(x) \\
 &= \cos(\answer[given]{1+2x})\cdot \answer[given]{2} \\
 &= 2\cos(1+2x)
\end{align*}
\end{explanation}
\end{example}
\end{document}